\chapter{Specifying and running a computation}

\section{Normalizations}
\label{sec:normalizations}

{\setlength{\parindent}{0cm}
Dimensional quantities in \sfincs~are normalized to ``reference'' values that are denoted by a bar:\\
$\bar{B}$ = reference magnetic field, typically 1 Tesla.\\
$\bar{R}$ = reference length, typically 1 meter.\\
$\bar{n}$ = reference density, typically $10^{19}$ m$^{-3}$, $10^{20}$ m$^{-3}$, or something similar.\\
$\bar{m}$ = reference mass, typically either the mass of hydrogen or deuterium.\\
$\bar{T}$ = reference temperature in energy units, typically 1 eV or 1 keV.\\
$\bar{v} = \sqrt{2 \bar{T} / \bar{m}}$ = thermal speed at the reference temperature and mass\\
$\bar{\Phi}$ = reference electrostatic potential, typically 1 V or 1 kV.\\
}

You can choose any reference parameters you like, not just the values
suggested here. However, if you use a {\ttfamily vmec} or {\ttfamily .bc} magnetic equilibrium
by choosing {\ttfamily geometryScheme} = 5, 11, or 12, then you MUST use $\bar{B}$ = 1 Tesla and $\bar{R}$ = 1 meter.
The code ``knows'' about the reference values only through
the 3 combinations {\ttfamily Delta}, {\ttfamily alpha}, and {\ttfamily nu\_n}
in the {\ttfamily physicsParameters} namelist.

Normalized quantities are denoted by a ``hat''.  Taking the magnetic field as an example,
$\hat{B}=B/\bar{B}$, where $\hat{B}$ is called {\ttfamily BHat} in the fortran code and \HDF~output file.

\section{Radial coordinates}
\label{sec:radialCoordinates}

A variety of flux-surface label coordinates are used in other codes and in the literature.
One common choice (used in \vmec) is $\psi_N$, the toroidal flux normalized to its
value at the last closed flux surface.  Another common choice is an ``effective normalized minor radius''
$r_N$, defined by $r_N=\sqrt{\psi_N}$.  For gradients of density, temperature, and electrostatic potential (i.e. the radial
electric field), it is useful to use a dimensional local minor radius $r = r_N a$, where $a$ is some
measure of the plasma effective outer minor radius.  Finally, one could also use $\psi$ directly.
For maximum flexibility, \sfincs~permits any of these four radial coordinates to be used, and different radial
coordinates can be used in different aspects of a given computation.  Output quantities which depend
on the radial coordinate, such as radial fluxes, are often given with respect to all radial coordinates.
In \sfincs, the four radial coordinates are named as follows:\\

{\setlength{\parindent}{0cm}

{\ttfamily psiHat} = $\hat\psi$ is the toroidal flux (divided by $2\pi$), normalized by $\bar{B}\bar{R}^2$.\\

{\ttfamily psiN} = $\psi_N$ is the toroidal flux normalized by its value at the last closed flux surface.\\

{\ttfamily rHat} = $\hat{r}$ is defined as {\ttfamily aHat}$\sqrt{\mbox{\ttfamily psiN}}$, where {\ttfamily aHat} is an effective minor radius of the last closed flux surface normalized by $\bar{R}$.\\

{\ttfamily rN} = $r_N$ is defined as $\sqrt{\mbox{\ttfamily psiN}}$.\\

}

These four radial coordinates are identified by the numbers 0, 1, 2, and 3 respectively.

When setting up a run, you can make several independent choices for radial coordinates.  One parameter
you select is {\ttfamily inputRadialCoordinateForGradients} in the {\ttfamily geometryParameters} namelist.
This parameter controls which coordinate is used to specify the gradients. The possible values of {\ttfamily inputRadialCoordinateForGradients} are:\\

{\setlength{\parindent}{0cm}

0: Use derivatives with respect to {\ttfamily psiHat}: Density gradients are specified using {\ttfamily dnHatdpsiHats}, 
temperature gradients are specified using {\ttfamily dTHatdpsiHats},  a single $E_r$ is specified using
{\ttfamily dPhiHatdpsiHat}, and a range of $E_r$ for a scan is specified using {\ttfamily dPhiHatdpsiHatMin}-{\ttfamily dPhiHatdpsiHatMax}.
\\

1: Use derivatives with respect to {\ttfamily psiN}: Density gradients are specified using {\ttfamily dnHatdpsiNs}, 
temperature gradients are specified using {\ttfamily dTHatdpsiNs},  a single $E_r$ is specified using
{\ttfamily dPhiHatdpsiN}, and a range of $E_r$ for a scan is specified using {\ttfamily dPhiHatdpsiNMin}-{\ttfamily dPhiHatdpsiNMax}.
\\

2: Use derivatives with respect to {\ttfamily rHat}: Density gradients are specified using {\ttfamily dnHatdrHats}, 
temperature gradients are specified using {\ttfamily dTHatdrHats},  a single $E_r$ is specified using
{\ttfamily dPhiHatdrHat}, and a range of $E_r$ for a scan is specified using {\ttfamily dPhiHatdrHatMin}-{\ttfamily dPhiHatdrHatMax}.
\\

3: Use derivatives with respect to {\ttfamily rN}: Density gradients are specified using {\ttfamily dnHatdrNs}, 
temperature gradients are specified using {\ttfamily dTHatdrNs},  a single $E_r$ is specified using
{\ttfamily dPhiHatdrN}, and a range of $E_r$ for a scan is specified using {\ttfamily dPhiHatdrNMin}-{\ttfamily dPhiHatdrNMax}.
\\

}

Another choice involving radial coordinates is how to specify the flux surface for the computation.
This choice is made using the parameter {\ttfamily inputRadialCoordinate} in the {\ttfamily geometryParameters}
namelist, which is again an integer from 0 to 3, and this parameter need not be the same as {\ttfamily inputRadialCoordinateForGradients}.
An extra complication with specifying the flux surface is that the magnetic equilibrium file will contain data on a finite number of surfaces,
and you may wish to use one of these surfaces.  For this reason, the parameters for specifying the flux surface have {\ttfamily \_wish}
appended to the name. In other words, the allowed values for {\ttfamily inputRadialCoordinate} are:\\

{\setlength{\parindent}{0cm}

0: Specify the flux surface using {\ttfamily psiHat\_wish}.\\

1: Specify the flux surface using {\ttfamily psiN\_wish}.\\

2: Specify the flux surface using {\ttfamily rHat\_wish}.\\

3: Specify the flux surface using {\ttfamily rN\_wish}.\\

}

When using {\ttfamily geometryScheme} == 11 or 12, \sfincs~will always shift the ``wish'' value so it matches an available surface in the magnetic equilibrium file.
For {\ttfamily geometryScheme} == 5, the {\ttfamily VMECRadialOption} parameter lets you can choose whether to shift to the nearest surface in the magnetic equilibrium file,
or to interpolate the \vmec~data onto the exact value of radius you specify.

If you perform a radial scan, then there is a third choice you can make: which radial coordinate to use in the {\ttfamily profiles} file.
This choice is made with an integer 0, 1, 2, or 3 in the first non-comment line of the {\ttfamily profiles} file.
The radial coordinate used in the {\ttfamily profiles} file need not be the same as either
{\ttfamily inputRadialCoordinate} or {\ttfamily inputRadialCoordinateForGradients}.
Note however that the maximum and minimum radial electric field specified in the {\ttfamily profiles}
file must be defined as the derivative of the electrostatic potential with respect to the radial coordinate {\ttfamily inputRadialCoordinateForGradients}.

For more details about the behavior of {\ttfamily inputRadialCoordinate}, {\ttfamily inputRadialCoordinateForGradients}, and {\ttfamily VMECRadialOption},
see section \ref{sec:geometryParameters}.



\section{Trajectory models}

As discussed in \cite{sfincsPaper},
one of the capabilities of \sfincs~is to compare various models for the terms in the kinetic equation involving $E_r$.
These variations of the kinetic equation are called ``trajectory models'' in \cite{sfincsPaper}.
The relevant terms in the kinetic equation can be turned off and on by certain Boolean parameters in the {\ttfamily physicsParameters} namelist.
The models described in \cite{sfincsPaper} are selected as follows:\\

{\setlength{\parindent}{0cm}

\underline{Full trajectories:}\\
{\ttfamily 
includeXDotTerm = .true.\\
includeElectricFieldTermInXiDot = .true.\\
useDKESExBDrift = .false.\\
}

\underline{Partial trajectories:}\\
{\ttfamily
includeXDotTerm = .false.\\
includeElectricFieldTermInXiDot = .false.\\
useDKESExBDrift = .false.\\
}

\underline{DKES trajectories:}\\
{\ttfamily
includeXDotTerm = .false.\\
includeElectricFieldTermInXiDot = .false.\\
useDKESExBDrift = .true.\\
}
}

There is not a significant difference in computational cost between these models.


\section{Quasineutrality and variation of the electrostatic potential on the flux surface}

One choice you should consider in setting up a computation is whether or not
to include variation on the flux surface of the electrostatic potential, $\Phi_1(\theta,\zeta)$.
Such variation does occur to some degree in a real plasma, but it is 
neglected in analytical theory and in many codes such as \dkes.  It can be proved
that including $\Phi_1$ has no effect on the particle or heat fluxes, parallel flows, or bootstrap current
when $E_r=0$, but generally there can be some difference when $E_r \ne 0$.
(There is a subtlety in showing that the heat flux is the same with and without $\Phi_1$,
discussed in the notes 20150325-01 in \path{sfincs/doc/}.)

In \sfincs, you can choose whether or not to
include $\Phi_1$ using the paramter {\ttfamily includePhi1} in the {\ttfamily physicsParameters}
namelist.  If and only if $\Phi_1$ is included, a quasineutrality equation is solved
at each point on the flux surface.  Due to these extra unknowns ($\Phi_1$) and extra equations
(quasineutrality), the system matrix is slightly larger when {\ttfamily includePhi1} is \true.
Specifically, the number of rows and columns are each increased by \Ntheta$\times$\Nzeta$+1$.  This increase is miniscule compared
to the number of rows and columns associated with the kinetic equation, which depends not only on real space
but also on velocity space and species.  Thus, there is very little extra computational cost associated
with {\ttfamily includePhi1}.  You may wish to set {\ttfamily includePhi1} = \true~when
using \sfincs~to model an experiment, and set  {\ttfamily includePhi1} = \false~when
comparing \sfincs~with analytic theory or with another code that does not include $\Phi_1$.


\section{Parallelization}
Choosing the number of nodes and procs.

\section{Issues with running on 1 processor}

\section{Monoenergetic transport coefficients}
\label{sec:monoenergetic}

By setting {\ttfamily RHSMode=3}, \sfincs~can be run in a mode
where it solves the same kinetic equation (prior to discretization) as {\ttfamily dkes}
and other monoenergetic codes.
When {\ttfamily RHSMode=3}, the values of {\ttfamily Zs}, {\ttfamily THats}, {\ttfamily nHats},
{\ttfamily mHats}, {\ttfamily nu\_n}, and {\ttfamily dPhiHatdXXX} are all ignored.
Instead, the collisionality is set by {\ttfamily nuPrime}, and the radial electric field is set
by {\ttfamily EStar}.  The first of these quantities is the dimensionless collisionality
\begin{equation}
\mbox{\ttfamily nuPrime} = \frac{(G+\iota I) \nu}{v B_0}
\end{equation}
where $G$ and $I$ are defined in (\ref{eq:covariant}), $\iota=1/q$ is the rotational transform,
$v$ is the speed at which the monoenergetic calculation is being performed, and $B_0$ is the (0,0) Fourier harmonic of $B$
with respect to the Boozer poloidal and toroidal angles. The collision
frequency $\nu$ is here the value of $\nu_\mathrm{ii}$ one would have if
$v$ were the thermal speed. That is, in SI units,
\begin{equation}
  \nu=\frac{4}{3\sqrt{\pi}}\frac{n Z^4e^4\ln \Lambda}{4\pi\epsilon_0^2m^2v^3}.
\end{equation}
%
The normalized radial electric field is
\begin{equation}
\mbox{\ttfamily EStar} = \frac{cG}{\iota v B_0} \frac{d\Phi}{d\psi}
\end{equation}
(Gaussian units).
When {\ttfamily RHSMode} == 1, {\ttfamily nuPrime} and {\ttfamily EStar} are ignored.
\todo{Should be change the behavior of RHSMode=2 so it uses nuPrime and EStar instead of nu\_n?}

The two parameters {\ttfamily nuPrime} and {\ttfamily EStar} are
related to the corresponding DKES parameters {\ttfamily CMUL} and
{\ttfamily EFIELD} by
\begin{eqnarray}
  \mbox{\ttfamily
    CMUL}&\equiv&\frac{\nu_\mathrm{D}}{v}=\frac{3\sqrt{\pi}}{4}\left(\mathrm{erf}(1)-\mathrm{Ch}(1)\right)
  \frac{B_0}{G+\iota I}\mbox{\ttfamily nuPrime},\\
  \mbox{\ttfamily EFIELD}&\equiv&-\left[\frac{d\Phi}{dr}\right]_\mathrm{DKES}\frac{1}{vB_0}=-\frac{\iota}{G}\left[\frac{d\Psi}{dr}\right]_\mathrm{DKES}\mbox{\ttfamily EStar},
\end{eqnarray}
where $\mathrm{Ch}$ is the Chandrasekhar function and $\nu_\mathrm{D}$
is the actual pitch-angle deflection frequency of the particle.
\todo{These expressions are in SI units. Should there be some
  factors of $c$ to adhere to the Gaussian standard used here?
  Probably not.}
