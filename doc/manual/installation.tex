\chapter{Installation}


\section{Requirements}
\label{sec:requirements}

To compile \sfincs~you need the \href{http://www.mcs.anl.gov/petsc/}{\PETSc} library (real version, as opposed to complex version) 
and the \href{https://www.hdfgroup.org/HDF5/}{\HDF} library.  Even if you will be running \sfincs~in parallel, you
only need the serial version of \HDF, not the parallel version.
\PETSc~is used for iterative solution of large linear and nonlinear systems of equations, and \HDF~is used for saving output.
We have developed and tested \sfincs~ with \PETSc~
versions 3.2 through 3.5.  The commands in \PETSc~often change from version to version, so future versions of \PETSc~may require modifications
to the \sfincs~source code.

Although \sfincs~can be run on a single processor, usually you want to run it in parallel.  
In this case, you need {\ttfamily MPI}, and you need at least one
of the two libraries \href{http://mumps-solver.org/}{\mumps}
or \href{http://crd-legacy.lbl.gov/~xiaoye/SuperLU/}{\superludist}.  (Note that \superludist is a parallel library which is different from the serial library {\ttfamily superlu}).
Both \mumps~and \superludist~are parallelized libraries for direct solution of large sparse linear systems,
which \sfincs~uses to factorize the preconditioning matrix (discussed in secton \ref{sec:gmres}).
\PETSc~has a built-in serial sparse direct linear solver,
but it sometimes gives an error that there is a ``zero pivot'' when \mumps~and \superludist~have no problem solving the system;
therefore you may want to use \mumps~or \superludist~even for serial runs.  In our experience,
\mumps~requires less memory and time than \superludist~for solving a given linear system.

If you want to load {\ttfamily VMEC wout} files in \netCDF~format, then you need the \netCDF~library.
This library is not required for loading ASCII-format {\ttfamily VMEC wout} files.  If you want to compile \sfincs~without
\netCDF, then edit \path{sfincs/fortran/version3/makefile} so that no value is assigned to {\ttfamily USE\_NETCDF}.

The plotting routines \sfincsPlot~and \sfincsScanPlot~require \python~2.X, {\ttfamily numpy}, {\ttfamily scipy}, and {\ttfamily matplotlib}.
These \python~libraries are not required by the core fortran part of \sfincs.

Although older {\ttfamily MATLAB} versions of \sfincs~are included in the \sfincs~repository,
{\ttfamily MATLAB} is not required for running the fortran version of \sfincs.

\section{Cloning the repository}

The source code for \sfincs~is hosted in a {\ttfamily git} repository at
\url{https://github.com/landreman/sfincs}.
You obtain the \sfincs~source code by cloning the repository. This requires several steps.

\begin{enumerate}
\item Create an account on \url{github.com}, and sign in to {\ttfamily github}.
\item Go to your account settings page, by clicking the wrench icon on the top right.
\item Click on ``SSH keys'' on the left, and add an SSH key for the computer you wish to use. To do this, you may wish to read see the ``generating SSH keys'' guide which is linked to from that page.
\item From a terminal command line in the computer you wish to use, enter\\
{\ttfamily git clone git@github.com:landreman/sfincs.git}\\
 to download the repository.
\end{enumerate}

Any time after you have cloned the repository in this way, you can download future updates to the code by entering {\ttfamily git pull} from any subdirectory within your local copy.

\section{Makefiles and environment variables}

To use \sfincs~ you must set the environment variable {\ttfamily SFINCS\_SYSTEM}.
(For example, using the {\ttfamily bash} shell on the {\ttfamily edison}
computer, you would type\\
{\ttfamily export SFINCS\_SYSTEM=edison}\\
 at the command line
or in your {\ttfamily .bashrc} startup script.)
This variable is used in two ways. 
First, {\ttfamily make} uses this variable to look for the 
appropriate makefile in the
\path{sfincs/fortran/version3/makefiles} directory.  Second, the {\ttfamily SFINCS\_SYSTEM} environment
variable is used by
\sfincsScan~to determine the command for submitting jobs to the system's queue.

You will probably want to add the directory \path{sfincs/fortran/version3/utils/} to your
path.  This directory contains the scripts for plotting output and running parameter scans.

\section{Setting up \sfincs~on a new system}
If you are setting up \sfincs~on a new system, one for which there is no file
\path{sfincs/fortran/version3/makefiles/makefile.XXX},
there are several things you need to do.

First, copy one of the existing makefiles, and edit it as appropriate.

Second, you will need to edit {\ttfamily utils/sfincsScan}. Look for the {\ttfamily if} block near the top
with sections for {\ttfamily sfincsSystem = edison},{\ttfamily hydra}, and {\ttfamily laptop}. Add an analogous block
for your system to set the command used to submit jobs, and a {\ttfamily nameJobFile} function.

Third, if you want {\ttfamily make test} to work (see section \ref{sec:maketest}),
you will need to create files {\ttfamily job.SFINCS\_SYSTEM}
for each example in the \path{sfincs/fortran/version3/examples/} directory that you want to include
in the tests.  You may be able to use the same {\ttfamily job.SFINCS\_SYSTEM} file for each example, but for the largest examples,
you may want to use different numbers of processes or different queues for different examples.

\section{Compiling}

If your system uses ``modules'', make sure you have loaded any required modules.
(Requirements are discussed in section \ref{sec:requirements}).  
There may be
instructions for the specific modules required on your system in the comments in the
appropriate makefile \\{\ttfamily sfincs/fortran/version3/makefiles/makefile.SFINCS\_SYSTEM} for your system.

Next, to compile, go to the directory {\ttfamily sfincs/fortran/version3/} and run {\ttfamily make}.

\section{Make test}
\label{sec:maketest}

To test that your \sfincs~executable is working, you can run {\ttfamily make test}
from the \path{sfincs/fortran/version3/} directory.  Doing so will run
\sfincs~for some or all of the examples in the {\ttfamily sfincs/fortran/version3/examples/} directories.
(The runs will be performed in series if no queueing system is available, otherwise the runs will all
be submitted to the queueing system.)
After each example completes, several of the output quantities (such as parallel flows and radial fluxes)
will be checked, using the
{\ttfamily tests\_small.py} or {\ttfamily tests\_large.py} script in the example's directory.

If you run {\ttfamily make retest} from the \path{sfincs/fortran/version3/} directory,
no new runs of \sfincs~will be performed, but the {\ttfamily tests\_small.py} or {\ttfamily tests\_large.py} script
will be run on any existing {\ttfamily sfincsOutput.h5} files in the \path{sfincs/fortran/version3/examples/} directories.
