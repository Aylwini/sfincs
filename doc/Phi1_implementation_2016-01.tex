
\documentclass[12pt]{article}
\usepackage{url} 
%\usepackage[dvips]{graphicx}
\usepackage[pdftex]{graphicx}
\usepackage[latin1]{inputenc}
\usepackage{amsmath}
\usepackage{amssymb}
\usepackage{fancyhdr}
\usepackage{bm}
\usepackage{float}
\usepackage{color}
\usepackage{wrapfig}
\usepackage{multicol}
\usepackage[colorlinks=true,
		linkcolor=red,
		citecolor=blue,
		urlcolor=blue]{hyperref}

\setlength {\parindent} { 10mm} 
\setlength{\textheight}{230mm} 
\setlength{\textwidth}{160mm} 
\setlength{\oddsidemargin}{0mm}
\setlength{\topmargin}{-10mm} 
% newcommands
\newcommand{\p}{\partial}
\newcommand{\g}[1]{\mbox{\boldmath $#1$}}
\newcommand{\vi}{\g V_{\! \! i}}
\newcommand{\ps}{Pfirsch-Schl\"{u}ter} 
\newcommand{\lp}{\left(}
\newcommand{\rp}{\right)}
\newcommand{\ca}[1]{\mbox{\cal $#1$}}
\newcommand{\be}{\begin{displaymath}}
\newcommand{\ee}{\end{displaymath}}
\newcommand{\bn}{\begin{equation}}
\newcommand{\en}{\end{equation}}
\newcommand{\mygtrsim}{\mathrel{\mbox{\raisebox{-1mm}{$\stackrel{>}{\sim}$}}}}
\newcommand{\mylsim}{\mathrel{\mbox{\raisebox{-1mm}{$\stackrel{<}{\sim}$}}}}
\newcommand{\vek}{\bf}
\newcommand{\ten}{\sf}
\newcommand{\bfm}[1]{\mbox{\boldmath$#1$}}
\newcommand{\lang}{\left\langle}
\newcommand{\rang}{\right\rangle}
\newcommand{\vo}[1]{\left|\begin {array}{l} \mbox{} \\ \mbox{} \\$#1$ \end
{array}\right .}  
\newcommand{\von}[2]{\left |\begin {array}{l}
\mbox{}\\$#1$\\$#2$ \end {array}\right .}
\newcommand{\simgt}{\:{\raisebox{-1.5mm}{$\stackrel
{\textstyle{>}}{\sim}$}}\:}
\newcommand{\simlt}{\:{\raisebox{-1.5mm}{$\stackrel
{\textstyle{<}}{\sim}$}}\:}
%\renewcommand {\baselinestretch} {1.67}
%\pagestyle{empty}
\newcommand{\todo}[1]{\textbf{\textcolor{red}{TODO: #1}}}
\newcommand{\remark}[1]{\textbf{\textcolor{red}{REMARK: #1}}}

\title{Implementation of $\Phi_1$ in SFINCS}

\pagestyle{fancy}
\fancyhead{}
\chead{Albert Moll�n %850227-2019
\\ Implementation of $\Phi_1$ in SFINCS}
\cfoot{\thepage}
\renewcommand{\headrulewidth}{1pt}
\renewcommand{\footrulewidth}{1pt}
\setlength{\headheight}{28pt}
\setlength{\footskip}{25pt}

\newcommand{\red}[1]{\textcolor{red}{#1}}
\newcommand{\mE}{\mathcal{E}}
\newcommand{\energy}{\mathcal{E}}
\newcommand{\mK}{\mathcal{K}}
\newcommand{\mN}{\mathcal{N}}
\newcommand{\mD}{\mathcal{D}}
\newcommand{\ord}{\mathcal{O}}
\newcommand{\Tpe}{T_\perp}
\newcommand{\Tpa}{T_\|}
\newcommand{\vpe}{v_\perp}
\newcommand{\vpa}{v_\|}
\newcommand{\kpe}{k_\perp}
\newcommand{\kpa}{k_\|}
\newcommand{\Bv}{\mathbf{B}}
\newcommand{\Ev}{\mathbf{E}}
\newcommand{\bv}{\mathbf{b}}
\newcommand{\vv}{\mathbf{v}}
\newcommand{\cd}{\cdot}
\newcommand{\na}{\nabla}
\newcommand{\btheta}{\bar{\theta}}
\newcommand{\phit}{\tilde{\phi}}
\newcommand{\oert}{\tilde{\omega}_{Er}}

\begin{document}
\titlepage

\maketitle

\section*{EUTERPE old equations vs new equations}
We want to modify the implementation of the old EUTERPE equations \cite{regana} in SFINCS 
to the new equations \cite{reganaArxiv}.

The old equations for the particle trajectories and the drift-kinetic equation are
\begin{align}
\dot{\bm{R}} & =  v_\| \bm{b} - \frac{\na \Phi_0 \times \bm{b}}{B}  \\
\dot{v}_\| & =  - \frac{q}{m} \bm{b} \cdot \na \Phi_1 - \mu \bm{b} \cdot \na B - \frac{v_\|}{B^2} \left(\bm{b} \times \na B\right) \cdot \na \Phi_0 \\
\dot{\mu} & =  0
\label{eq:ParticleTrajEuterpeOld}
\end{align}
and
\begin{multline}
\frac{\p f_1}{\p t} + \dot{\bm{R}} \cdot \na f_1 + \dot{v}_\| \frac{\p f_1}{\p v_\|} - C = \\ =
- f_M \left[\frac{1}{n} \frac{\p n}{\p \psi} + \left(\frac{m v^2}{2 T} - \frac{3}{2}\right) \frac{1}{T} \frac{\p T}{\p \psi}\right] 
\left(\bm{v}_d + \bm{v}_{E1}\right) \cdot \na \psi - 
\frac{q}{m} \frac{f_M}{v_{\mathrm{th}}^2} \left(v_\| \bm{b} + \bm{v}_d\right) \cdot \left(\na \Phi_0 + \na \Phi_1\right).
\label{eq:DriftKineticEuterpeOld}
\end{multline}

The new equations are
\begin{align}
\dot{\bm{R}} & =  v_\| \bm{b} - \frac{\na \Phi_0 \times \bm{b}}{B}  \\
\dot{v}_\| & =  - \frac{q}{m} \bm{b} \cdot \na \Phi_1 - \mu \bm{b} \cdot \na B - \frac{v_\|}{B^2} \left(\bm{b} \times \na B\right) \cdot \na \Phi_0 \\
\dot{\mu} & =  0
\label{eq:ParticleTrajEuterpeNew}
\end{align}
and
\begin{multline}
\frac{\p f_1}{\p t} + \dot{\bm{R}} \cdot \na f_1 + \dot{v}_\| \frac{\p f_1}{\p v_\|} - C = \\ =
- f_0 \left[\frac{1}{n} \frac{\p n}{\p \psi} + \frac{q}{T} \frac{\p \Phi_0}{\p \psi} + \left(\frac{m v^2}{2 T} - \frac{3}{2} + \frac{q}{T} \Phi_1\right) \frac{1}{T} \frac{\p T}{\p \psi}\right] 
\left(\bm{v}_d + \bm{v}_{E1}\right) \cdot \na \psi.
\label{eq:DriftKineticEuterpeNew}
\end{multline}
Here we have the definitions
\begin{equation}
\Phi\left(\psi, \theta, \varphi\right) \equiv \Phi_0\left(\psi\right) + \Phi_1\left(\theta, \varphi\right),
\label{eq:Phi}
\end{equation}
\begin{equation}
\bm{v}_d = \frac{m}{q} \frac{\mu B + v_\|^2}{B^2} \bm{b} \times \na B,
\label{eq:MagneticDrift}
\end{equation}
\begin{equation}
\bm{v}_{E1} = - \frac{\na \Phi_1 \times \bm{b}}{B},
\label{eq:ElectricDrift1}
\end{equation}
\begin{equation}
f_0 = f_M \exp \left(- q \Phi_1 / T \right) = \frac{n_0\left(\psi\right)}{\left(2 \pi\right)^{3/2} v_{\mathrm{th}}^3} \exp \left[- \frac{\left(v_\|^2  + v_\perp^2\right)}{2 v_{\mathrm{th}}^2}\right] \exp \left(- q \Phi_1 / T \right),
\label{eq:f0}
\end{equation}
$q = Z e$ and $v_{\mathrm{th}}^2 = T/m$.\\

\noindent The only differences appear in the RHS:s of Eqs.~\ref{eq:DriftKineticEuterpeOld} and \ref{eq:DriftKineticEuterpeNew}:\\ 
Firstly, $f_M$ has been replaced by $f_0$ containing the $\exp \left(- q \Phi_1 / T \right)$ factor.\\ 
Secondly, some of the terms have been modified. 
We rewrite the RHS of \ref{eq:DriftKineticEuterpeNew}:
\begin{multline}
{\mathrm{RHS}}_{\mathrm{NEW}} =
- f_0 \left[\frac{1}{n} \frac{\p n}{\p \psi} + \frac{q}{T} \frac{\p \Phi_0}{\p \psi} + \left(\frac{m v^2}{2 T} - \frac{3}{2} + \frac{q}{T} \Phi_1\right) \frac{1}{T} \frac{\p T}{\p \psi}\right] 
\left(\bm{v}_d + \bm{v}_{E1}\right) \cdot \na \psi = \\ =
%%
- f_0 \left[\frac{1}{n} \frac{\p n}{\p \psi} + \frac{q}{T} \frac{\p \Phi_0}{\p \psi} + \left(\frac{m v^2}{2 T} - \frac{3}{2}\right) \frac{1}{T} \frac{\p T}{\p \psi}\right] 
\left(\bm{v}_d + \bm{v}_{E1}\right) \cdot \na \psi + \\ - 
f_0  \frac{q}{T} \Phi_1 \frac{1}{T} \frac{\p T}{\p \psi}
\left(\bm{v}_d + \bm{v}_{E1}\right) \cdot \na \psi = \\ =
%%
- f_0 \left[\frac{1}{n} \frac{\p n}{\p \psi} + \frac{q}{T} \frac{\p \Phi_0}{\p \psi} + \left(\frac{m v^2}{2 T} - \frac{3}{2}\right) \frac{1}{T} \frac{\p T}{\p \psi}\right] 
\bm{v}_d \cdot \na \psi + \\ - 
f_0 \left[\frac{1}{n} \frac{\p n}{\p \psi} + \left(\frac{m v^2}{2 T} - \frac{3}{2}\right) \frac{1}{T} \frac{\p T}{\p \psi}\right] 
\bm{v}_{E1} \cdot \na \psi - f_0 \frac{q}{T} \frac{\p \Phi_0}{\p \psi} \bm{v}_{E1} \cdot \na \psi + \\ -
f_0 \frac{q}{T} \Phi_1 \frac{\na T}{T} \cdot
\left(\bm{v}_d + \bm{v}_{E1}\right) = \\ =
%%
- f_0 \left[\frac{1}{n} \frac{\p n}{\p \psi} + \frac{q}{T} \frac{\p \Phi_0}{\p \psi} + \left(\frac{m v^2}{2 T} - \frac{3}{2}\right) \frac{1}{T} \frac{\p T}{\p \psi}\right] 
\bm{v}_d \cdot \na \psi + \\  
\textcolor{blue}{- f_0 \left[\frac{1}{n} \frac{\p n}{\p \psi} + \left(\frac{m v^2}{2 T} - \frac{3}{2}\right) \frac{1}{T} \frac{\p T}{\p \psi}\right] 
\bm{v}_{E1} \cdot \na \psi}  + \\  
\textcolor{red}{- f_0 \frac{q}{T} \left[\na \Phi_0 \cdot \bm{v}_{E1} + \Phi_1 \frac{\na T}{T} \cdot \bm{v}_d + \Phi_1 \frac{\na T}{T} \cdot \bm{v}_{E1} \right]}.
\label{eq:DriftKineticEuterpeNewRHS}
\end{multline}
Similarly, the RHS of \ref{eq:DriftKineticEuterpeOld} is rewritten as:
\begin{multline}
{\mathrm{RHS}}_{\mathrm{OLD}} =
- f_M \left[\frac{1}{n} \frac{\p n}{\p \psi} + \left(\frac{m v^2}{2 T} - \frac{3}{2}\right) \frac{1}{T} \frac{\p T}{\p \psi}\right] 
\left(\bm{v}_d + \bm{v}_{E1}\right) \cdot \na \psi - 
\frac{q}{m} \frac{f_M}{v_{\mathrm{th}}^2} \left(v_\| \bm{b} + \bm{v}_d\right) \cdot \left(\na \Phi_0 + \na \Phi_1\right) = \\ =
%%
- f_M \left[\frac{1}{n} \frac{\p n}{\p \psi} + \frac{q}{T} \frac{\p \Phi_0}{\p \psi} + \left(\frac{m v^2}{2 T} - \frac{3}{2}\right) \frac{1}{T} \frac{\p T}{\p \psi}\right] 
\bm{v}_d \cdot \na \psi + \\  
\textcolor{blue}{- f_M \left[\frac{1}{n} \frac{\p n}{\p \psi} + \left(\frac{m v^2}{2 T} - \frac{3}{2}\right) \frac{1}{T} \frac{\p T}{\p \psi}\right] 
\bm{v}_{E1} \cdot \na \psi}  + \\  
\textcolor{red}{- f_M \frac{q}{T} \left[v_\| \bm{b} \cdot  \na \Phi_1 + \bm{v}_d \cdot  \na \Phi_1\right]}.
\label{eq:DriftKineticEuterpeOldRHS}
\end{multline}
Comparing ${\mathrm{RHS}}_{\mathrm{NEW}}$ to ${\mathrm{RHS}}_{\mathrm{OLD}}$ we see that, apart from $f_M \rightarrow f_0$, only the terms in red have changed. 

\subsection*{Implementation in SFINCS}
The only part of the drift-kinetic equation block we need to modify is the RHS, where we need to update the red terms and substitute $f_M \rightarrow f_0$. SFINCS had earlier neglected the $\bm{v}_d \cdot  \na \Phi_1$-term which is small in the standard $\rho_\ast$-expansion. 
The RHS that was implemented is (see Matt's ISHW poster, also note that $\bm{v}_{E} \cdot \na \psi = \bm{v}_{E1} \cdot \na \psi$)
\begin{multline}
{\mathrm{RHS}}_{\mathrm{SFINCS, OLD}} =
- f_M \left[\frac{1}{n} \frac{\p n}{\p \psi} + \frac{q}{T} \frac{\p \Phi_0}{\p \psi} + \left(\frac{m v^2}{2 T} - \frac{3}{2}\right) \frac{1}{T} \frac{\p T}{\p \psi}\right] 
\bm{v}_d \cdot \na \psi + \\  
\textcolor{blue}{- f_M \left[\frac{1}{n} \frac{\p n}{\p \psi} + \left(\frac{m v^2}{2 T} - \frac{3}{2}\right) \frac{1}{T} \frac{\p T}{\p \psi}\right] 
\bm{v}_{E} \cdot \na \psi}  %+ \\  
\textcolor{red}{- f_M \frac{q}{T} v_\| \bm{b} \cdot  \na \Phi_1 }.
\label{eq:DriftKineticSFINCSOldRHS}
\end{multline}
We thus replace 
\begin{equation}
\textcolor{red}{
v_\| \bm{b} \cdot  \na \Phi_1
}
\label{eq:ReplaceOLD}
\end{equation}
with 
\begin{equation}
\textcolor{red}{
\na \Phi_0 \cdot \bm{v}_{E} + \Phi_1 \frac{\na T}{T} \cdot \bm{v}_d + \Phi_1 \frac{\na T}{T} \cdot \bm{v}_{E}},
\label{eq:ReplaceNEW}
\end{equation}
and make the substitution
\begin{equation}
\textcolor{red}{
f_M \rightarrow f_0 = f_M \exp \left(- q \Phi_1 / T \right)}.
\label{eq:substitution}
\end{equation}
\remark{In EUTERPE $\Phi_1$ is only an unknown in the quasi-neutrality equation, in the kinetic equation it is an input which means that there are no nonlinearities. It also means that the exponential in $f_0$ is not expanded in the kinetic equation. 
Are all terms in Eq.~\ref{eq:ReplaceNEW} feasible to implement in SFINCS? E.g. is it a problem that the $\Phi_1 \frac{\na T}{T} \cdot \bm{v}_{E}$-term contains 3 factors with $\Phi_1$?}\\

\noindent \remark{Because of $f_0$ it seems as all terms which contain $\Phi_1$ are now nonlinear. Does it make sense to have both switches \textbf{includePhi1} and \textbf{nonlinear} available in SFINCS?}
%%%%%%%%%%%%%%%%%%%%%%%%%%%%%%%%%%%%%%%%%%%%%%%%%%%%%%%%%%%%%%%%%%%%%%%%%%%%%%%%%%%%%%%%%%%%%%%%%%%%%%%%%%%%%%%%%%%%%%%%%%%%%%%%%%%%%
\newpage
\section*{Quasi-neutrality equation}
In EUTERPE $\Phi_1$ is calculated from quasi-neutrality by expanding the exponential, assuming adiabatic electrons and neglecting the impurities: 
\begin{equation}
n_s = n_{s 0} \left(\psi\right) \, \exp \left(- q_s \Phi_1 / T_s \right) + n_{s 1},
\label{eq:}
\end{equation}
\begin{equation}
\sum_s Z_s n_s = 0,
\label{eq:Quasineutrality}
\end{equation}
\[
\Rightarrow \;\;\; 0 \simeq \sum_s Z_s \left[n_{s 0} \left(1 - q_s \Phi_1 / T_s \right) + n_{s 1}\right] \;\;\; \Leftrightarrow \;\;\;
\]
\[
\sum_s Z_s \left[n_{s 0} + n_{s 1}\right] = \sum_s \frac{Z_s^2 e}{T_s} \Phi_1 n_{s 0} .
\]
Since $n_{s 0} \left(\psi\right)$ is obtained by integrating the Maxwellian $f_{M s}$ over velocity space we must have 
\[
\sum_s Z_s n_{s 0} = 0,
\]
which yields 
\begin{equation}
\sum_s Z_s n_{s 1} - \Phi_1 \sum_s \frac{Z_s^2 e}{T_s} n_{s 0}  = 0.
\label{eq:Quasineutrality2}
\end{equation}
With kinetic ions, adiabatic electrons ($n_{e 1} = 0$) and neglecting impurities we obtain 
\begin{equation}
\Phi_1 = \frac{T_e}{e} \left[\frac{T_e}{T_i} n_{i 0} + n_{e 0}\right]^{-1} n_{i 1}.
\label{eq:Phi1}
\end{equation}

\subsection*{Implementation in SFINCS}
For a first benchmark, we want to implement the same equations as EUTERPE in SFINCS. \\
\remark{This is not a very generic quasi-neutrality equation so it is possible that we might want to change it in SFINCS later.} \\
In the code we add an adiabatic species which only enters into the quasi-neutrality equation, and neglect its collisional impact on the kinetic species (the effect of ion-electron collisions is small compared to ion-ion collisions). 
Moreover, we will only consider the first of the kinetic species in quasi-neutrality and neglect the rest. 
This is implemented by modifying the LHS of the row corresponding to quasi-neutrality in the block-matrix structure of Matt's ISHW poster, adding the adiabatic term to the $\Phi_1$-column and removing all kinetic species except the first. \\
\remark{It feels a bit weird to remove species from quasi-neutrality, even if the impurity density is small. Does this mean that we should removed the check that the input densities are quasi-neutral and instead check that $n_{i 0} \left(\psi\right) = n_{e 0} \left(\psi\right)$ in the input?}\\
The equation we will implement in SFINCS is thus 
\begin{equation}
Z_i n_{i 1} - \Phi_1 \left[ \frac{Z_i^2 e}{T_i} n_{i 0} + \frac{Z_e^2 e}{T_e} n_{e 0}\right] = 0.
\label{eq:QuasineutralityIonElectrons}
\end{equation}
We note that
\begin{multline}
n_s = n_{s 0} \left(\psi\right) \, \exp \left(- q_s \Phi_1 / T_s \right) + n_{s 1} = \int d^3 v f_{Ms} \exp \left(- q_s \Phi_1 / T_s \right) + \int d^3 v f_{1 s} = \\ = 
d^3 v f_{0 s} + d^3 v f_{1 s}.
\label{eq:distributionToDensity}
\end{multline}
The velocity integration is SFINCS is done in $\left(x, \xi\right) = \left(v / v_s, v_\| / v\right)$, and 
\begin{equation}
\int d^3 v = 2 \pi v_{s}^3 \int_{0}^{\infty} dx\, x^2 \int_{-1}^{1} d\xi
\label{eq:VelocityIntegration}
\end{equation}
(note that $v_{s}^2 = 2 T_s/m_s$ differs from Jose's notation $v_{\mathrm{th}}^2 = T/m$). 
Using SFINCS normalizations $n_s = \bar{n}  \hat{n}_s$, $T_s = \bar{T}  \hat{T}_s$, $v_s / \bar{v} = \sqrt{\hat{T}_s / \hat{m}_s}$, 
$f_s = \bar{n} \hat{f}_s / \bar{v}^3$, 
we find 
\begin{equation}
\hat{n}_s = 2 \pi \left(\hat{T}_s / \hat{m}_s\right)^{3/2} \int_{0}^{\infty} dx\, x^2 \int_{-1}^{1} d\xi \hat{f}_s.
\label{eq:VelocityIntegrationSFINCS}
\end{equation}
Also using $\Phi_1 = \bar{\Phi} \hat{\Phi}_1$ and $\alpha = e \bar{\Phi} / \bar{T}$ we can write Eq.~\ref{eq:QuasineutralityIonElectrons} 
\begin{equation}
Z_i \hat{n}_{i 1} - \alpha \hat{\Phi}_1 \left[ \frac{Z_i^2}{\hat{T}_i} \hat{n}_{i 0} + \frac{Z_e^2}{\hat{T}_e} \hat{n}_{e 0}\right] = 0
\label{eq:QuasineutralityIonElectronsSFINCS}
\end{equation}
and finally obtain 
\begin{equation}
\left[2 \pi Z_i \left(\hat{T}_i / \hat{m}_i\right)^{3/2} \int_{0}^{\infty} dx\, x^2 \int_{-1}^{1} d\xi \hat{f}_{i 1}\right]
- \alpha \hat{\Phi}_1 \left[ \frac{Z_i^2}{\hat{T}_i} \hat{n}_{i 0} + \frac{Z_e^2}{\hat{T}_e} \hat{n}_{e 0}\right] = 0.
\label{eq:QuasineutralityIonElectronsSFINCS2}
\end{equation}
This is the equation we will implement in the code, but adding a $\lambda$ to make the system square.\\ 
\remark{Is the $2\pi$ factor correct in Eq.~\ref{eq:QuasineutralityIonElectronsSFINCS2}? It is not in the former implementation of quasi-neutrality, but in that situation it could be divided away.}
%%%%%%%%%%%%%%%%%%%%%%%%%%%%%%%%%%%%%%%%%%%%%%%%%%%%%%%%%%%%%%%%%%%%%%%%%%%%%%%%%%%%%%%%%%%%%%%%%%%%%%%%%%%%%%%%%%%%%%%%%%%%%%%%%%%%%
\newpage


%%%%%%%%%%%%%%%%%%%%%%%%%%%%%%%%%%%%%%%%%%%%%%%%%%%%%%%%%%%%%%%%%%%%%%%%%%%%%%%%%%%%%%%%%%%%%%%%%%%%%%%%%%%%%%%%%%%%%%%%%%%%%%%%%%%%%

\begin{thebibliography}{99}

\bibitem{regana} J.~M.~Garc\'{\i}a-Rega\~{n}a, R.~Kleiber, C.~D.~Beidler, Y.~Turkin, H.~Maa{\ss}berg  
and P.~Helander, 
\href{http://dx.doi.org/10.1088/0741-3335/55/7/074008}{\em Plasma Phys.~Control.~Fusion} {\bf 55} (2013) 074008.

\bibitem{reganaArxiv} J.~M.~Garc\'{\i}a-Rega\~{n}a, C.~D.~Beidler, Y.~Turkin, R.~Kleiber, P.~Helander, H.~Maa{\ss}berg, J. A. Alonso and J. L. Velasco,  
\href{http://arxiv.org/abs/1501.03967}{\em arXiv:1501.03967} (2015).

\bibitem{landremanSFINCS} Landreman~M, Smith~H~M, Moll\'en~A and Helander~P 2014
 \href{http://dx.doi.org/10.1063/1.4870077}{\em Phys. Plasmas} {\bf 21} 042503

%\bibitem{simakov} A.~N.~Simakov, P.~Helander,
%  {\em Phys. Plasmas} {\bf 16}, 042503 (2009).
  
%\bibitem{nonAxis} P.~Helander, Theory of plasma confinement in non-axisymmetric magnetic fields (2013).

%\bibitem{MH} L.~R�de and B.~Westergren, Mathematics Handbook for Science and Engineering, $5^{\mathrm{th}}$ edition, 2004. %\vspace{-5mm}

%\bibitem{Abra} M.~Abramowitz and I.~A.~Stegun, Handbook of Mathematical Functions, $10^{\mathrm{th}}$ printing, 1972.

\end{thebibliography}

\end{document}