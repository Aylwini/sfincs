\chapter{Specifying and running a computation}

\section{Normalizations}

{\setlength{\parindent}{0cm}
Dimensional quantities in \sfincs~are normalized to ``reference'' values that are denoted by a bar:\\
$\bar{B}$ = reference magnetic field, typically 1 Tesla.\\
$\bar{R}$ = reference length, typically 1 meter.\\
$\bar{n}$ = reference density, typically $10^{19}$ m$^{-3}$, $10^{20}$ m$^{-3}$, or something similar.\\
$\bar{m}$ = reference mass, typically either the mass of hydrogen or deuterium.\\
$\bar{T}$ = reference temperature in energy units, typically 1 eV or 1 keV.\\
$\bar{v} = \sqrt{2 \bar{T} / \bar{m}}$ = reference speed\\
$\bar{\Phi}$ = reference electrostatic potential, typically 1 V or 1 kV.\\
}

You can choose any reference parameters you like, not just the values
suggested here. However, if you use a {\ttfamily vmec} or {\ttfamily .bc} magnetic equilibrium
by choosing {\ttfamily geometryScheme} = 5, 11, or 12, then you MUST use $\bar{B}$ = 1 Tesla and $\bar{R}$ = 1 meter.
The code ``knows'' about the reference values only through
the 3 combinations {\ttfamily Delta}, {\ttfamily alpha}, and {\ttfamily nu\_n}
in the {\ttfamily physicsParameters} namelist.

Normalized quantities are denoted by a ``hat''.  Taking the magnetic field as an example,
$\hat{B}=B/\bar{B}$, where $\hat{B}$ is called {\ttfamily BHat} in the fortran code and \HDF~output file.

\section{Radial coordinates}
\label{sec:radialCoordinates}

A variety of flux-surface label coordinates are used in other codes and in the literature.
One common choice (used in {\ttfamily vmec}) is $\psi_N$, the toroidal flux normalized to its
value at the last closed flux surface.  Another common choice is an ``effective normalized minor radius''
$r_N$, defined by $r_N=\sqrt{\psi_N}$.  For gradients of density, temperature, and electrostatic potential (i.e. the radial
electric field), it is useful to use a dimensional local minor radius $r = r_N a$, where $a$ is some
measure of the plasma effective outer minor radius.  Finally, one could also use $\psi$ directly.
For maximum flexibility, \sfincs~permits any of these four radial coordinates to be used, and different radial
coordinates can be used in different parts of a given computation.  Output quantities which depend
on the radial coordinate, such as radial fluxes, are often given with respect to all radial coordinates.

To specify a run, you can make several independent choices for radial coordinates.  One parameter
you select is {\ttfamily inputRadialCoordinate} in the {\ttfamily geometryParameters} namelist.
This parameter controls which coordinate is used to specify the flux surface. The possible values of {\ttfamily inputRadialCoordinate} are:

0: Use {\ttfamily psiHat}, the toroidal flux (divided by $2\pi$) normalized by $\bar{B}\bar{R}^2$.\\

1: Use {\ttfamily psiN}, the toroidal flux normalized by its value at the last closed flux surface.\\

2: Use {\ttfamily rHat}, defined as {\ttfamily aHat}$\sqrt{\mbox{\ttfamily psiN}}$, where {\ttfamily aHat} is an effective minor radius of the last closed flux surface normalized by $\bar{R}$.\\

3: Use {\ttfamily rN}, defined as $\sqrt{\mbox{\ttfamily psiN}}$.\\





\section{Parallelization}
Choosing the number of nodes and procs.

\section{Issues with running on 1 processor}
