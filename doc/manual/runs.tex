\chapter{Specifying and running a computation}

\section{Normalizations}
\label{sec:normalizations}

{\setlength{\parindent}{0cm}
Dimensional quantities in \sfincs~are normalized to ``reference'' values that are denoted by a bar:\\
$\bar{B}$ = reference magnetic field, typically 1 Tesla.\\
$\bar{R}$ = reference length, typically 1 meter.\\
$\bar{n}$ = reference density, typically $10^{20}$ m$^{-3}$, $10^{19}$ m$^{-3}$, or something similar.\\
$\bar{m}$ = reference mass, typically either the mass of hydrogen or deuterium.\\
$\bar{T}$ = reference temperature in energy units, typically 1 eV or 1 keV.\\
$\bar{v} = \sqrt{2 \bar{T} / \bar{m}}$ = thermal speed at the reference temperature and mass\\
$\bar{\Phi}$ = reference electrostatic potential, typically 1 V or 1 kV.\\
}

You can choose any reference parameters you like, not just the values
suggested here. However, if you use a {\ttfamily vmec} or {\ttfamily .bc} magnetic equilibrium
by choosing \parlink{geometryScheme} = 5, 11, or 12, then you MUST use $\bar{B}$ = 1 Tesla and $\bar{R}$ = 1 meter.
The code ``knows'' about the reference values only through
the 3 combinations \parlink{Delta}, \parlink{alpha}, and \parlink{nu\_n}
in the {\ttfamily \hyperref[sec:physicsParameters]{physicsParameters}} namelist.

Normalized quantities are denoted by a ``hat''.  Taking the magnetic field as an example,
$\hat{B}=B/\bar{B}$, where $\hat{B}$ is called {\ttfamily BHat} in the fortran code and \HDF~output file.

\section{Radial coordinates}
\label{sec:radialCoordinates}

A variety of flux-surface label coordinates are used in other codes and in the literature.
One common choice (used in \vmec) is $\psi_N$, the toroidal flux normalized to its
value at the last closed flux surface.  Another common choice is an ``effective normalized minor radius''
$r_N$, defined by $r_N=\sqrt{\psi_N}$.  For gradients of density, temperature, and electrostatic potential (i.e. the radial
electric field), it is useful to use a dimensional local minor radius $r = r_N a$, where $a$ is some
measure of the plasma effective outer minor radius.  Finally, one could also use $\psi$ directly.
For maximum flexibility, \sfincs~permits any of these four radial coordinates to be used, and different radial
coordinates can be used in different aspects of a given computation.  Output quantities which depend
on the radial coordinate, such as radial fluxes, are often given with respect to all radial coordinates.
In \sfincs, the four radial coordinates are named as follows:\\

{\setlength{\parindent}{0cm}

{\ttfamily \hypertarget{psiHat}{psiHat}} = $\hat\psi$ is the toroidal flux (divided by $2\pi$), normalized by $\bar{B}\bar{R}^2$.\\

{\ttfamily \hypertarget{psiN}{psiN}} = $\psi_N$ is the toroidal flux normalized by its value at the last closed flux surface.\\

{\ttfamily \hypertarget{rHat}{rHat}} = $\hat{r}$ is defined as {\ttfamily aHat}$\sqrt{\mbox{\ttfamily psiN}}$, where {\ttfamily aHat} is an effective minor radius of the last closed flux surface normalized by $\bar{R}$.\\

{\ttfamily \hypertarget{rN}{rN}} = $r_N$ is defined as $\sqrt{\mbox{\ttfamily psiN}}$.\\

}

These four radial coordinates are identified by the numbers 0, 1, 2, and 3 respectively.

When setting up a run, you can make several independent choices for radial coordinates.  One parameter
you select is \parlink{inputRadialCoordinateForGradients} in the 
{\ttfamily \hyperref[sec:geometryParameters]{geometryParameters}} namelist.
This parameter controls which coordinate is used to specify the gradients. The possible values of \parlink{inputRadialCoordinateForGradients} are:\\

{\setlength{\parindent}{0cm}

0: Use derivatives with respect to {\ttfamily psiHat}: Density gradients are specified using \parlink{dnHatdpsiHats}, 
temperature gradients are specified using \parlink{dTHatdpsiHats},  a single $E_r$ is specified using
\parlink{dPhiHatdpsiHat}, and a range of $E_r$ for a scan is specified using \parlink{dPhiHatdpsiHatMin}-\parlink{dPhiHatdpsiHatMax}.
\\

1: Use derivatives with respect to {\ttfamily psiN}: Density gradients are specified using \parlink{dnHatdpsiNs}, 
temperature gradients are specified using \parlink{dTHatdpsiNs},  a single $E_r$ is specified using
\parlink{dPhiHatdpsiN}, and a range of $E_r$ for a scan is specified using \parlink{dPhiHatdpsiNMin}-\parlink{dPhiHatdpsiNMax}.
\\

2: Use derivatives with respect to {\ttfamily rHat}: Density gradients are specified using \parlink{dnHatdrHats}, 
temperature gradients are specified using \parlink{dTHatdrHats},  a single $E_r$ is specified using
\parlink{dPhiHatdrHat}, and a range of $E_r$ for a scan is specified using \parlink{dPhiHatdrHatMin}-\parlink{dPhiHatdrHatMax}.
\\

3: Use derivatives with respect to {\ttfamily rN}: Density gradients are specified using \parlink{dnHatdrNs}, 
temperature gradients are specified using \parlink{dTHatdrNs},  a single $E_r$ is specified using
\parlink{dPhiHatdrN}, and a range of $E_r$ for a scan is specified using \parlink{dPhiHatdrNMin}-\parlink{dPhiHatdrNMax}.
\\

}

Another choice involving radial coordinates is how to specify the flux surface for the computation.
This choice is made using the parameter \parlink{inputRadialCoordinate} in the {\ttfamily \hyperref[sec:geometryParameters]{geometryParameters}}
namelist, which is again an integer from 0 to 3, and this parameter need not be the same as \parlink{inputRadialCoordinateForGradients}.
An extra complication with specifying the flux surface is that the magnetic equilibrium file will contain data on a finite number of surfaces,
and you may wish to use one of these surfaces.  For this reason, the parameters for specifying the flux surface have {\ttfamily \_wish}
appended to the name. In other words, the allowed values for \parlink{inputRadialCoordinate} are:\\

{\setlength{\parindent}{0cm}

0: Specify the flux surface using \parlink{psiHat\_wish}.\\

1: Specify the flux surface using \parlink{psiN\_wish}.\\

2: Specify the flux surface using \parlink{rHat\_wish}.\\

3: Specify the flux surface using \parlink{rN\_wish}.\\

}

When using \parlink{geometryScheme} == 11 or 12, \sfincs~will always shift the ``wish'' value so it matches an available surface in the magnetic equilibrium file.
For \parlink{geometryScheme} == 5, the \parlink{VMECRadialOption} parameter lets you can choose whether to shift to the nearest surface in the magnetic equilibrium file,
or to interpolate the \vmec~data onto the exact value of radius you specify.

If you perform a radial scan, then there is a third choice you can make: which radial coordinate to use in the {\ttfamily profiles} file.
This choice is made with an integer 0, 1, 2, or 3 in the first non-comment line of the {\ttfamily profiles} file.
The radial coordinate used in the {\ttfamily profiles} file need not be the same as either
\parlink{inputRadialCoordinate} or \parlink{inputRadialCoordinateForGradients}.
Note however that the maximum and minimum radial electric field specified in the {\ttfamily profiles}
file must be defined as the derivative of the electrostatic potential with respect to the radial coordinate \parlink{inputRadialCoordinateForGradients}.

For more details about the behavior of \parlink{inputRadialCoordinate}, \parlink{inputRadialCoordinateForGradients}, and \parlink{VMECRadialOption},
see section \ref{sec:geometryParameters}.

\section{Trajectory models}

As discussed in \cite{sfincsPaper},
one of the capabilities of \sfincs~is to compare various models for the terms in the kinetic equation involving $E_r$.
These variations of the kinetic equation are called ``trajectory models'' in \cite{sfincsPaper}.
The relevant terms in the kinetic equation can be turned off and on by certain Boolean parameters in the {\ttfamily \hyperref[sec:physicsParameters]{physicsParameters}} namelist.
The models described in \cite{sfincsPaper} are selected as follows:\\

{\setlength{\parindent}{0cm}

\underline{Full trajectories:}\\
{\ttfamily 
\parlink{includeXDotTerm} = .true.\\
\parlink{includeElectricFieldTermInXiDot} = .true.\\
\parlink{useDKESExBDrift} = .false.\\
}

\underline{Partial trajectories:}\\
{\ttfamily
\parlink{includeXDotTerm} = .false.\\
\parlink{includeElectricFieldTermInXiDot} = .false.\\
\parlink{useDKESExBDrift} = .false.\\
}

\underline{DKES trajectories:}\\
{\ttfamily
\parlink{includeXDotTerm} = .false.\\
\parlink{includeElectricFieldTermInXiDot} = .false.\\
\parlink{useDKESExBDrift} = .true.\\
}
}

There is not a significant difference in computational cost between these models.


\section{Quasineutrality and variation of the electrostatic potential on the flux surface}

One choice you should consider in setting up a computation is whether or not
to include variation on the flux surface of the electrostatic potential, $\Phi_1(\theta,\zeta)$.
Such variation does occur to some degree in a real plasma, but it is 
neglected in analytical theory and in many codes such as \dkes.  It can be proved
that including $\Phi_1$ has no effect on the particle or heat fluxes, parallel flows, or bootstrap current
when $E_r=0$, but generally there can be some difference when $E_r \ne 0$.
(There is a subtlety in showing that the heat flux is the same with and without $\Phi_1$,
discussed in the notes 20150325-01 in \path{sfincs/doc/}.)

In \sfincs, you can choose whether or not to
include $\Phi_1$ using the paramter {\ttfamily \hyperlink{includePhi1}{includePhi1}} in the 
{\ttfamily \hyperref[sec:physicsParameters]{physicsParameters}}
namelist.  If and only if $\Phi_1$ is included, a quasineutrality equation is solved
at each point on the flux surface.  Due to these extra unknowns ($\Phi_1$) and extra equations
(quasineutrality), the system matrix is slightly larger when \parlink{includePhi1} is \true.
Specifically, the number of rows and columns are each increased by \Ntheta$\times$\Nzeta$+1$.  This increase is miniscule compared
to the number of rows and columns associated with the kinetic equation, which depends not only on real space
but also on velocity space and species.  Thus, there is very little extra computational cost associated
with \parlink{includePhi1}.  You may wish to set \parlink{includePhi1} = \true~when
using \sfincs~to model an experiment, and set \parlink{includePhi1} = \false~when
comparing \sfincs~with analytic theory or with another code that does not include $\Phi_1$.

\section{Poloidal and toroidal magnetic drifts}

You can choose to either include or not include the poloidal and toroidal magnetic drifts.
These drifts are turned off by default.  To turn them on, all you need to do is
set \parlink{magneticDriftScheme} = 1 in the {\ttfamily \hyperref[sec:physicsParameters]{physicsParameters}} namelist.
If on, you must use VMEC
geometry (\parlink{geometryScheme} = 5).  The magnetic drift terms introduce nonzeros in the system matrix,
and therefore increase the memory required for factorization; a typical increase is 30-40\%.

\section{Parallelization}
[Coming soon...] Choosing the number of nodes and procs.

\section{Issues with running on 1 processor}

\section{Monoenergetic transport coefficients}
\label{sec:monoenergetic}

By setting \parlink{RHSMode} = 3, \sfincs~can be run in a mode
where it solves the same kinetic equation (prior to discretization) as 
\dkes~and other monoenergetic codes.
When \parlink{RHSMode} = 3, the values of \parlink{Zs}, \parlink{THats}, \parlink{nHats},
\parlink{mHats}, \parlink{nu\_n}, and {\ttfamily dPhiHatdXXX} are all ignored.
Instead, the collisionality is set by \parlink{nuPrime}, and the radial electric field is set
by \parlink{EStar}.  The first of these quantities is the dimensionless collisionality
\begin{equation}
\mbox{\parlink{nuPrime}} = \frac{(G+\iota I) \nu}{v B_0}
\end{equation}
where $G$ and $I$ are defined in (\ref{eq:covariant}), $\iota=1/q$ is the rotational transform,
$v$ is the speed at which the monoenergetic calculation is being performed, and $B_0$ is the (0,0) Fourier harmonic of $B$
with respect to the Boozer poloidal and toroidal angles. The collision
frequency $\nu$ is here the value of $\nu_\mathrm{ii}$ one would have if
$v$ were the thermal speed. That is, in SI units,
\begin{equation}
  \nu=\frac{4}{3\sqrt{\pi}}\frac{n Z^4e^4\ln \Lambda}{4\pi\epsilon_0^2m^2v^3}.
\end{equation}
%
The normalized radial electric field is
\begin{equation}
\mbox{\parlink{EStar}} = \frac{cG}{\iota v B_0} \frac{d\Phi}{d\psi}
\end{equation}
(Gaussian units).
When {\ttfamily RHSMode} == 1, {\ttfamily nuPrime} and {\ttfamily EStar} are ignored.
\todo{Should be change the behavior of RHSMode=2 so it uses nuPrime and EStar instead of nu\_n?}

The two parameters \parlink{nuPrime} and \parlink{EStar} are
related to the corresponding DKES parameters {\ttfamily CMUL} and
{\ttfamily EFIELD} by
\begin{eqnarray}
  \mbox{\ttfamily
    CMUL}&\equiv&\frac{\nu_\mathrm{D}}{v}=\frac{3\sqrt{\pi}}{4}\left(\mathrm{erf}(1)-\mathrm{Ch}(1)\right)
  \frac{B_0}{G+\iota I}\mbox{\ttfamily nuPrime},\\
  \mbox{\ttfamily EFIELD}&\equiv&-\left[\frac{d\Phi}{dr}\right]_\mathrm{DKES}\frac{1}{vB_0}=-\frac{\iota}{G}\left[\frac{d\Psi}{dr}\right]_\mathrm{DKES}\mbox{\ttfamily EStar},
\end{eqnarray}
where $\mathrm{Ch}$ is the Chandrasekhar function and $\nu_\mathrm{D}$
is the actual pitch-angle deflection frequency of the particle.
\todo{These expressions are in SI units. Should there be some
  factors of $c$ to adhere to the Gaussian standard used here?
  Probably not.}

\section{Poloidal and toroidal angles}

If you are interested in any of the output quantities that vary on a flux surface,
such as the density or electrostatic potential, then it is important to know
how the poloidal and toroidal angles ($\theta$ and $\zeta$) in \sfincs~are defined.
The definitions of the poloidal and toroidal angles in 
\sfincs~depend on the input parameter \parlink{geometryScheme}. When a \vmec~equilibrium is imported by setting
\parlink{geometryScheme} = 5, then \sfincs~will use the same poloidal and toroidal angles
as \vmec.  The toroidal angle in this case is the normal cylindrical coordinate. Note that field lines
are not straight in these \vmec~coordinates.
For any other setting of \parlink{geometryScheme}, \sfincs~will use Boozer coordinates.

