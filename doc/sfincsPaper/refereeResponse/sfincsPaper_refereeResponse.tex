\documentclass[11pt]{article}
\usepackage[dvips,usenames,dvipsnames]{color}
\usepackage[dvips]{graphicx}
\usepackage{delarray}
\usepackage[latin1]{inputenc}
\usepackage{amsmath}
\usepackage{amssymb}
\setlength {\parindent} { 10mm} 
\setlength{\textheight}{240mm} 
\setlength{\textwidth}{160mm} 
\setlength{\oddsidemargin}{0mm}
\setlength{\topmargin}{-10mm} 


% newcommands
\newcommand{\vect}[1]{\mbox{\boldmath $#1$}}
\newcommand{\todo}[1]{{\color{red}#1}}
\newenvironment{referee}{\begin{quote}\it\color{Blue}}{\end{quote}}

\begin{document}  
\section*{Response to the referee}

Re: ``Comparison of particle trajectories and collision operators for
collisional transport in nonaxisymmetric plasmas'', by
M Landreman et al., manuscript POP42544.\newline


Thank you for your careful reading of the manuscript and constructive suggestions.
Below we address each of your points in detail.

\begin{referee}
If it is not difficult numerically to retain the poloidal and toroidal magnetic drifts in the
equations of motion, as it is written in the middle of page 13, why do authors [not] keep these term
and remove only the $v_{ma}\cdot\nabla\psi$ component from eqs.(12) and (14)? I understand that it is to make
the ``DKES trajectories'', but it seems that approximating in (12) and (14) still
holds the conservation of $\mu$. Is it just because ``The omitted terms may be important in other
situations, but here our primary interest is the treatment of the $d\Phi/d\psi$ terms'' that the magnetic
drift term is completely dropped from (12) and (14), or is there any other unexplained problems
in solving the transport equation with the poloidal and toroidal magnetic drift terms kept?
\end{referee}

We do not expect any fundamental complications to arise if the poloidal and toroidal magnetic drifts
were retained in the kinetic equation. 
However, we have several reasons for neglecting these magnetic drifts
 in this manuscript.
\begin{enumerate}
\item The ratio of these magnetic drifts to the parallel streaming term we keep is $\sim \rho_* = \rho / L$
 (the gyroradius to scale length), so neglecting the poloidal and toroidal magnetic drifts is
equivalent to considering the $\rho_* \to 0$ limit.
Since our primary interest in this paper is to compare the various options for the electric field
terms and collision operator, it is reasonable to take this $\rho_* \to 0$ limit for simplicity.
\item
Retaining these drifts would make the transport matrix (and figures 1, 2, 4, and 4)
depend on $\rho_*$ in addition to $\nu'$ and $E_*$. 
The dependence of the matrix elements on $\nu'$ and $E_*$ is already  complicated
and interesting to discuss in the $\rho_* \to 0$ limit,
without the extra complexity of finite-$\rho_*$ effects.
%\item Including these magnetic drift terms would also introduce dependence on the radial
%derivatives of the magnetic field
\item The poloidal and toroidal magnetic drifts would add extra complicating
terms to the analysis of moments of the kinetic equation in section III.
%\item If the poloidal and toroidal magnetic drifts are retained without the radial magnetic drift, ????
%\item The poloidal and magnetic drift terms will take additional time to implement in the code, 
%and the 
\end{enumerate}

We have added some text to section II (between (15)-(16)) and a sentence to section V (between \todo{(35)-(36)}) to clarify these issues.

\begin{referee}
Here the conservation properties of mass and energy in three trajectory models are discussed,
but how about the parallel momentum balance? Relating to this question, in page 20 - 21
authors have discussed about the relation between the momentum balance and friction force
from different collision operator models. I agree with the authors that the momentum
conservation property in collision operator is important to evaluate correct transport coefficients.
However, by seeing the figures 3(c)(d), parallel momentum balance seems to be broken in the
partial trajectory model, even if the full linearized Fokker-Planck operator is used.
The momentum balance equation (36) is derived by taking the $\vect{b}\cdot \int d^3v\, m\vect{v}$ moment of the drift
kinetic equation (15). Then, for each three trajectory model, taking the $\vect{b}\cdot \int d^3v\, m\vect{v}$ moment of
eq. (15) with combining one of eq. (17),(18), or (19), does it yield the proper parallel momentum
balance equation as eq.(36)? Or more simply, do these three trajectory models evaluate the
parallel viscosity $\left<\vect{b}\cdot\nabla\cdot\vect{\pi}\right>$ correctly?
\end{referee}

We have added a new section IV to the paper to discuss momentum balance.
To summarize, we first discuss how parallel momentum has a different status to mass and energy in that mass and
energy are conserved, whereas parallel momentum is not conserved due to the mirror force.
We then compare the $m v_{||}$ moments of the various forms of the drift-kinetic
equation to the full fluid momentum equation, as you suggest.
We show the electric field terms in the kinetic perspective
are associated with a piece of the gyroviscosity in the fluid perspective.  The full trajectory model
gives the correct viscous force whereas the other trajectory models do not.
One further noteworthy property of the parallel velocity moments of the kinetic equations
is that of the trajectory models considered, only the full trajectories preserve intrinsic
ambipolarity in quasisymmetry when $E_r \ne 0$. This result is now discussed at the end of section IV.

\begin{referee}
I do not understand why the source term (25) does not work well with the pitch-angle scattering
operator or P.-A. scattering + momentum conservation operator. It is not the difference in the
collision operator but difference in the trajectory model that is related to the conservation
property of energy, (22). If the original motivation of introducing the term is to maintain the
mass and energy conservation property for finite-Er, then why you need to change the form of
source/sink term if you choose P.-A. scattering operator, which actually conserves the energy,
while you do not need according to the trajectory model? Does it mean diffusion process in
v-coordinate is essential for the source/sink term to work properly?
Please explain the reason of changing the source term according to the collision operator more
detail.
\end{referee}

We have expanded the discussion at the end of section III to clarify this issue.

The system of sources and extra constraint equations actually serves two independent purposes,
not only to circumvent the conservation problems when $E_r$ is nonzero, but also to eliminate the null
space of the kinetic equation. This latter problem arises even for the DKES trajectories 
and even at $E_r=0$.  
When $E_r \ne 0$, the sources are needed to eliminate the conservation problems, and the extra constraints 
can be thought of as a convenient way to keep the linear system square (number of equations = number of unknowns) upon discretizatiton.
Even when $E_r = 0$, and even for the DKES trajectories in which sources are not required, the constraints are needed to eliminate
the null space in the kinetic equation, and the source terms can be thought of as
a convenient way to keep the linear system square upon discretization.
Any particle and heat source will be sufficient to solve the first problem (conservation), but to solve the second problem (null space),
the number of constraints should match the dimensionality of the null space.  This dimension is set by the collision operator (2 for
Fokker-Planck, $N_x$ for the other two operators.)
This is why the source/constraint scheme is chosen based on the collision operator.

\begin{referee}
Concerning the source/sink term, I think that the usage of such an artificial term is relevant if
the effect of the term is small. Can you show the effect of the source/sink term is small or not in
some cases shown in Section VI, by comparing $\left< \int d^3v\, S(\psi,v) \right> \Delta t$ and $n(\psi)$, for example, where
$\Delta t$ is some characteristic time scale such as collision time or transport time scale.
\end{referee}

The sources normalized to the transport timescale are now plotted in figure 3.
As discussed in the new text accompanying this figure, 
the sources normalized by the ion collision time are only a factor $1/0.4=2.5$ larger than the values plotted,
so the source terms remain much smaller than the collision operator in the kinetic equation.

\begin{referee}
Explanation for Fig 1 and 2: Though authors write ``As $E_r \to 0$, all the matrix elements converge
smoothly to their $E_r = 0$ limits'', since the horizontal axis of these figures are in log scale, we
cannot judge if their claim is true or not. Please show the value of $L_{jk}(E_r=0)$ in these figures,
too.
\end{referee}

Computations for $E_r=0$ are now plotted on figures (1)-(2), as $\blacktriangleright$ symbols on the left axis
of each sub-figure.

\begin{referee}
It is estimated that $E_* \sim 0.1 - 0.01$ in most of the plasma. However, if the core $T_e$ is higher than
$T_i$ and ambipolar $E_r$ is positive (electron root) in ECH-heated plasma, $E_*$ for ions easily becomes
nearly, or even larger than, unity near the magnetic axis, since $\rho_\theta \propto 1/B_p$ is large and usually
$|E_r|$(ion root) $\ll E_r$(electron root). See Yokoyama et al., Nuclear Fusion (2007), 1213 for example.
I think therefore the statement ``In most of the plasma, $E_*$ is however expected to be smaller
than a few percent.'' is too strong and may be misleading.
\end{referee}

We agree that in scenarios with $T_e \gg T_i$, such as a Core Electron Root Confinement regime,
$E_*$ may be larger than a few percent. (Our original statement
was meant only to apply to a standard ion-root W7-X regime,
but this should have been clarified.) The text in the manuscript has now been updated to acknowledge
the possibility of an electron root and that $E_*$ may not be small. 

\begin{referee}
Appendix: The authors claims that the parallel flow and heat flux solved by SFINCS code obeys the
isomorphism, but is radial electric field ($d\Phi/dr$) contained in ``other input quantities''? Please
clarity that. Also, if the isomorphism is satisfied for finite $d\Phi/dr$ and finite source term, I
suppose that the usage of source term in the full trajectory model is justified at least in
quasisymmetric systems. Can author comment on this point in Appendix?
\end{referee}

We have changed and expanded the text in this appendix for clarification. 
For all of the trajectory models described, an isomorphism does hold for the transport matrix elements
if the electric field is varied so
$(GM+IN)(\iota M-N)^{-1} d\Phi_0/d\psi$ remains constant as $M$ and $N$ are varied.
%This required transformation of $d\Phi_0/d\psi$ can be found in the last line of table 1 in Ref. [49].

The second part of your question is an interesting observation: when the full trajectory model
is applied to a quasisymmetric field, the source vanishes for any $E_r$, not just at a special ambipolar $E_r$ (as in a non-quasisymmetric field.)
We now state this fact in the new discussion of intrinsic ambipolarity at the end of section IV.

\vspace{0.3in}


In addition to the aforementioned changes, the following modifications have been made to the manuscript:

\begin{itemize}
\item Albert Moll\'{e}n has recently demonstrated precise agreement between SFINCS and an anlytical calculation
of Simakov and Helander for the limit of high collisionality [43]. His results are displayed in a new figure 6, with details given in Appendix B,
 and he has been included as a co-author.
\item Output from the DKES code is now plotted in figure 5, to demonstrate the good agreement with SFINCS.
\end{itemize}


Sincerely yours,\newline

M. Landreman

H. M. Smith

A. Moll\'{e}n

P. Helander

\end{document}
