\chapter{Input Parameters}
\label{ch:input}

\newcommand{\param}[5]{{\setlength{\parindent}{0cm} {\ttfamily \bfseries #1}\\{\it Type}: #2\\{\it Default}: #3\\{\it When it matters}: #4\\{\it Meaning}: #5}}
\newcommand{\ssparam}[4]{{\setlength{\parindent}{0cm} {\ttfamily \bfseries #1}\\{\it Type}: #2\\{\it When it matters}: #3\\{\it Meaning}: #4}}
\newcommand{\PETScParam}[2]{{\setlength{\parindent}{0cm} {\ttfamily \bfseries #1}\\{\it Meaning}: #2}}
\newcommand{\myhrule}{{\setlength{\parindent}{0cm} \hrulefill }}

\newcommand{\true}{{\ttfamily .true.}}
\newcommand{\false}{{\ttfamily .false.}}

In this chapter we first describe all the parameters which can be included in the {\ttfamily input.namelist} file. 
Then we list some of the command-line flags associated with \PETSc~which can be useful.
Note that all parameters in {\ttfamily input.namelist}, both for \sfincs~and \sfincsScan,
are case-insensitive.




\section{The {\ttfamily general} namelist}

The default values are usually best for the parameters in this namelist.

\myhrule

\param{RHSMode}
{integer}
{1}
{Always}
{Option related to the number of right-hand sides (i.e. inhomogeneous drive terms) for which the kinetic equation is solved.\\

{\ttfamily RHSMode=1}: Solve for a single right-hand side.\\

{\ttfamily RHSMode=2}: Solve for 3 right-hand sides to get the 3x3 transport matrix. Presently implemented only for 1 species.\\

{\ttfamily RHSMode=3}: Solve for the 2x2 monoenergetic transport coefficients. When this option is chosen, {\ttfamily Nx} is set to 1 and only 1 species is used.
}

\myhrule

\param{outputFileName}
{string}
{``sfincsOutput.h5''}
{Always}
{Name which will be used for the HDF5 output file.  If this parameter is changed from the default value, \sfincsScan~ will not work.}

\myhrule

\param{saveMatlabOutput}
{Boolean}
{\false}
{Always}
{If this switch is set to true, Matlab m-files are created which
 store the system matrix, right-hand side, and solution vector.  If an iterative solver is used,
 the preconditioner matrix is also saved.
 PETSc usually generates an error message if you ask to save Matlab output when
the size of the linear system is more then 1400 $\times$ 1400,
 so usually this setting should be false except for very small test problems.
}

\myhrule

\param{MatlabOutputFilename}
{string}
{``sfincsMatrices''}
{Only when {\ttfamily saveMatlabOutput == .true.}}
{Start of the filenames which will be used for Matlab output.}

\myhrule

\param{saveMatricesAndVectorsInBinary}
{Boolean}
{\false}
{Always}
{If this switch is set to true, the matrix, right-hand-side, and solution of the
linear system will be saved in PETSc's binary format.  The preconditioner matrix will also
be saved if {\ttfamily tryIterativeSolver == .true.}
}

\myhrule

\param{binaryOutputFilename}
{string}
{``sfincsBinary''}
{Only when {\ttfamily saveMatricesAndVectorsInBinary == .true.}}
{Start of the filenames which will be used for binary output.}

\myhrule

\param{solveSystem}
{Boolean}
{\true}
{Always}
{If this parameter is false, the system of equations will not actually be solved.
 Sometimes it can be useful to set this parameter to {\ttfamily .false.} when debugging.
}








\section{The {\ttfamily geometryParameters} namelist}
\label{sec:geometryParameters}

The parameters in this namelist define the magnetic geometry, and so you will almost certainly want to modify some of these parameters.

\myhrule

\param{geometryScheme}
{integer}
{1}
{Always}
{How the magnetic geometry is specified.\\

{\ttfamily geometryScheme}==1: Use the following 3-helicity model:\\
\begin{eqnarray}
\label{eq:Bmodel}
B(\theta,\zeta)/\bar{B} 
&=& (\mbox{\ttfamily B0OverBBar}) [1 + (\mbox{\ttfamily epsilon\_t}) \cos(\theta) \\
&& + (\mbox{\ttfamily epsilon\_h}) \cos((\mbox{\ttfamily helicity\_l}) \theta - (\mbox{\ttfamily helicity\_n}) \zeta) \nonumber \\
&&  + (\mbox{\ttfamily epsilon\_antisymm}) \nonumber \\
&& \times \sin((\mbox{ \ttfamily helicity\_antisymm\_l}) \theta - (\mbox{\ttfamily helicity\_antisymm\_n}) \zeta)] \nonumber
\end{eqnarray}

(All the variables in this formula are discussed later in this namelist.)\\

{\ttfamily geometryScheme}==2: Use a 3-helicity model of the LHD standard configuration at {\ttfamily rN}=0.5.\\

{\ttfamily geometryScheme}==3: Use a 4-helicity model of the LHD inward-shifted configuration at {\ttfamily rN}=0.5.\\

{\ttfamily geometryScheme}==4: Use a 3-helicity model of the W7-X standard configuration at {\ttfamily rN}=0.5.\\

{\ttfamily geometryScheme}==5: Read the {\ttfamily vmec wout} file specified in {\ttfamily equilibriumFile} below. The file can be
       either ASCII format or \netCDF~format. (\sfincs~will auto-detect the format.).\\

{\ttfamily geometryScheme}==11: Read the IPP {\ttfamily .bc} format Boozer-coordinate file specified in {\ttfamily equilibriumFile} below. 
The file is assumed to be stellarator-symmetric.\\

{\ttfamily geometryScheme}==12: Read the IPP {\ttfamily .bc} format Boozer-coordinate file specified in {\ttfamily equilibriumFile} below. 
The file is assumed to be stellarator-\emph{asymmetric}.
}

\myhrule

\param{inputRadialCoordinate}
{integer}
{3}
{When {\ttfamily geometryScheme} == 1, 5, 11, or 12}
{Which radial coordinate to use to specify the flux surface for a single calculation,
or to specify the range of flux surfaces for a radial scan. 
(Regardless of the value of this parameter, when {\ttfamily geometryScheme} == 2, 3, or 4, the flux surface used will be {\ttfamily rN}=0.5.)
See section \ref{sec:radialCoordinates}
for more information about radial coordinates.\\

{\ttfamily inputRadialCoordinate}==0: Use the flux surface specified by {\ttfamily psiHat\_wish}
for a single run, and use the range specified by {\ttfamily psiHat\_min} and {\ttfamily psiHat\_max} for radial scans.\\

{\ttfamily inputRadialCoordinate}==1: Use the flux surface specified by {\ttfamily psiN\_wish}
for a single run, and use the range specified by {\ttfamily psiN\_min} and {\ttfamily psiN\_max} for radial scans.\\

{\ttfamily inputRadialCoordinate}==2: Use the flux surface specified by {\ttfamily rHat\_wish}
for a single run, and use the range specified by {\ttfamily rHat\_min} and {\ttfamily rHat\_max} for radial scans.\\

{\ttfamily inputRadialCoordinate}==3: Use the flux surface specified by {\ttfamily rN\_wish}
for a single run, and use the range specified by {\ttfamily rN\_min} and {\ttfamily rN\_max} for radial scans. \\

No matter which option you pick, the value of all 4 radial coordinates used will be saved in the output \HDF~file.
}

\myhrule

\param{inputRadialCoordinateForGradients}
{integer}
{2}
{Whenever {\ttfamily RHSMode}==1.}
{Which radial coordinate is used to use to specify the input gradients of density, temperature, and electrostatic potential,
i.e. which radial coordinate is used in the denominator of these derivatives.
See section \ref{sec:radialCoordinates}
for more information about radial coordinates.\\

{\ttfamily inputRadialCoordinateForGradients}==0: Density gradients are specified by {\ttfamily dnHatdpsiHats},
temperature gradients are specified by {\ttfamily dTHatdpsiHats}, a single $E_r$ is specified by {\ttfamily dPhiHatdpsiHat},
and the range of an $E_r$ scan is specified by {\ttfamily dPhiHatdpsiHatMin-dPhiHatdpsiHatMax}.\\

{\ttfamily inputRadialCoordinateForGradients}==1: Density gradients are specified by {\ttfamily dnHatdpsiNs},
temperature gradients are specified by {\ttfamily dTHatdpsiNs}, a single $E_r$ is specified by {\ttfamily dPhiHatdpsiN},
and the range of an $E_r$ scan is specified by {\ttfamily dPhiHatdpsiNMin-dPhiHatdpsiNMax}.\\

{\ttfamily inputRadialCoordinateForGradients}==2: Density gradients are specified by {\ttfamily dnHatdrHats},
temperature gradients are specified by {\ttfamily dTHatdrHats}, a single $E_r$ is specified by {\ttfamily dPhiHatdrHat},
and the range of an $E_r$ scan is specified by {\ttfamily dPhiHatdrHatMin-dPhiHatdrHatMax}.\\

{\ttfamily inputRadialCoordinateForGradients}==3: Density gradients are specified by {\ttfamily dnHatdrNs},
temperature gradients are specified by {\ttfamily dTHatdrNs}, a single $E_r$ is specified by {\ttfamily dPhiHatdrN},
and the range of an $E_r$ scan is specified by {\ttfamily dPhiHatdrNMin-dPhiHatdrNMax}.\\

No matter which option you pick, the gradients with respect to all 4 radial coordinates will be saved in the output \HDF~file.
}

\myhrule

\param{psiHat\_wish}
{real}
{-1}
{Only when {\ttfamily inputRadialCoordinate} == 0 and {\ttfamily geometryScheme} == 1, 5, 11, or 12.}
{Requested flux surface for the computation.  See section \ref{sec:radialCoordinates}
for more information about radial coordinates.}

\myhrule

\param{psiN\_wish}
{real}
{0.25}
{Only when {\ttfamily inputRadialCoordinate} == 1 and {\ttfamily geometryScheme} == 1, 5, 11, or 12.}
{Requested flux surface for the computation.  See section \ref{sec:radialCoordinates}
for more information about radial coordinates.}

\myhrule

\param{rHat\_wish}
{real}
{-1}
{Only when {\ttfamily inputRadialCoordinate} == 2 and {\ttfamily geometryScheme} == 1, 5, 11, or 12.}
{Requested flux surface for the computation.  See section \ref{sec:radialCoordinates}
for more information about radial coordinates.}

\myhrule

\param{rN\_wish}
{real}
{0.5}
{Only when {\ttfamily inputRadialCoordinate} == 3 and {\ttfamily geometryScheme} == 1, 5, 11, or 12.}
{Requested flux surface for the computation.  See section \ref{sec:radialCoordinates}
for more information about radial coordinates.}

\myhrule

\param{B0OverBBar}
{real}
{1.0}
{Only when {\ttfamily geometryScheme} == 1. Otherwise, {\ttfamily B0OverBBar} will be set according to the requested {\ttfamily geometryScheme}.}
{Magnitude of the (0,0) Boozer harmonic of the magnetic field strength (equivalent to $\left< B^3\right>/\left<B^2\right>$), normalized by $\bar{B}$.}

\myhrule

\param{GHat}
{real}
{3.7481}
{Only when {\ttfamily geometryScheme} == 1. Otherwise, {\ttfamily GHat} will be set according to the requested {\ttfamily geometryScheme}.}
{$G$ is $(c/2)\times$ the poloidal current outside the flux
 surface. Equivalently, $G$ is the coefficient of $\nabla\zeta_B$ in the
 covariant representation of $\vect{B}$ in terms of Boozer coordinates $(\theta_B,\zeta_B)$:
\begin{equation}
\label{eq:covariant}
\vect{B}(\psi,\theta_B,\zeta_B) = \beta(\psi,\theta_B,\zeta_B)\nabla\psi + I(\psi)\nabla\theta_B + G(\psi)\nabla\zeta_B.
\end{equation}
{\ttfamily GHat} is $G$ normalized by $\bar{B}\bar{R}$.}

\myhrule

\param{IHat}
{real}
{0.0}
{Only when {\ttfamily geometryScheme} == 1. Otherwise, {\ttfamily IHat} will be set according to the requested {\ttfamily geometryScheme}.}
{$I$ is $(c/2)\times$ the toroidal current inside the flux
 surface. Equivalently, $I$ is the coefficient of $\nabla\theta_B$ in the
 covariant representation of $\vect{B}$ in terms of Boozer coordinates $(\theta_B,\zeta_B)$ in (\ref{eq:covariant}).
{\ttfamily IHat} is $I$ normalized by $\bar{B}\bar{R}$.}

\myhrule

\param{iota}
{real}
{0.4542}
{Only when {\ttfamily geometryScheme} == 1. Otherwise, {\ttfamily iota} will be set according to the requested {\ttfamily geometryScheme}.}
{Rotational transform (rationalized), equivalent to $1/q$ where $q$ is the safety factor.}

\myhrule

\param{epsilon\_t}
{real}
{-0.07053}
{Only when {\ttfamily geometryScheme} == 1.}
{Toroidal variation in $B$, as defined by (\ref{eq:Bmodel}).}

\myhrule

\param{epsilon\_h}
{real}
{0.05067}
{Only when {\ttfamily geometryScheme} == 1.}
{Helical variation in $B$, as defined by (\ref{eq:Bmodel}).}

\myhrule

\param{epsilon\_antisymm}
{real}
{0.0}
{Only when {\ttfamily geometryScheme} == 1.}
{Stellarator-antisymmetric variation in $B$, as defined by (\ref{eq:Bmodel}).}

\myhrule

\param{helicity\_l}
{integer}
{2}
{Only when {\ttfamily geometryScheme} == 1.}
{Poloidal mode number of the helical variation in $B$, as defined by (\ref{eq:Bmodel}).}

\myhrule

\param{helicity\_n}
{integer}
{10}
{Only when {\ttfamily geometryScheme} == 1.}
{Toroidal mode number of the helical variation in $B$, as defined by (\ref{eq:Bmodel}).}

\myhrule

\param{helicity\_antisymm\_l}
{integer}
{1}
{Only when {\ttfamily geometryScheme} == 1.}
{Poloidal mode number of the stellarator-antisymmetric variation in $B$, as defined by (\ref{eq:Bmodel}).}

\myhrule

\param{helicity\_antisymm\_n}
{integer}
{0}
{Only when {\ttfamily geometryScheme} == 1.}
{Toroidal mode number of the stellarator-antisymmetric variation in $B$, as defined by (\ref{eq:Bmodel}).
Note that you can create an up-down asymmetric tokamak by setting {\ttfamily helicity\_antisymm\_n}=0, {\ttfamily epsilon\_h}=0,
and {\ttfamily epsilon\_antisymm}$>$0.}

\myhrule

\param{psiAHat}
{real}
{0.15596}
{Only when {\ttfamily geometryScheme} == 1. Otherwise, {\ttfamily psiAHat} will be set according to the requested {\ttfamily geometryScheme}.}
{{\ttfamily psiAHat} = $\psi_a / (\bar{B} \bar{R}^2)$
where $2 \pi \psi_a$ is the toroidal flux at the last closed flux surface.}

\myhrule

\param{aHat}
{real}
{0.5585}
{Only when {\ttfamily geometryScheme} == 1. Otherwise, {\ttfamily aHat} will be set according to the requested {\ttfamily geometryScheme}.}
{The effective minor radius at the last closed flux surface, in units of $\bar{R}$.
The code only uses {\ttfamily aBar} for converting between the various radial coordinates in input and output quantities.
}

\myhrule

\param{equilibriumFile}
{string}
{``''}
{Only when {\ttfamily geometryScheme} == 5, 11, or 12.}
{Filename from which to load the magnetic equilibrium, either in {\ttfamily vmec wout} ASCII or \netCDF~format, or IPP {\ttfamily .bc} format.}

\myhrule

\param{VMECRadialOption}
{integer}
{1}
{Only when {\ttfamily geometryScheme} == 5.}
{Controls whether the nearest available flux surface in the {\ttfamily vmec wout} file is used, or whether radial interpolation is applied to the \vmec~data
to obtain the magnetic field components on the exact surface requested.\\

{\ttfamily VMECRadialOption}=0: Use the exact {\ttfamily XXX\_wish} flux surface requested, by interpolating from the \vmec~radial grid.\\

{\ttfamily VMECRadialOption}=1: Use a surface that may be slightly different from {\ttfamily XXX\_wish} to get the nearest available flux surface from \vmec's HALF grid.  
   The components of $\vect{B}$ in \vmec~are stored on the half grid, so interpolation is then unnecessary.\\

{\ttfamily VMECRadialOption}=2: Use a surface that may be slightly different from {\ttfamily XXX\_wish} to get the nearest available flux surface from \vmec's FULL grid.
I'm not sure why you would want this, but the feature is implemented for completeness.
}

\myhrule

\param{min\_Bmn\_to\_load}
{real}
{0.0}
{Only when {\ttfamily geometryScheme} == 5, 11, or 12.}
{Filters the magnetic field read from an input file.  Only Fourier modes $(m,n)$ for which $B_{m,n}$ is at least {\ttfamily min\_Bmn\_to\_load} will be included.}




\section{The {\ttfamily speciesParameters}~namelist}

This namelist defines which species are included in the calculation, along with the density and temperature and gradients thereof.
You will definitely want to set the parameters in this namelist.
Note that only one of the four parameters 
{\ttfamily dnHatdpsiHats}, 
{\ttfamily dnHatdpsiNs}, 
{\ttfamily dnHatdrHats}, or
{\ttfamily dnHatdrNs}
will be used, depending on the value of {\ttfamily inputRadialCoordinateForGradients}
in the {\ttfamily geometryParameters} namelist.
Similarly, only one of the four parameters
{\ttfamily dTHatdpsiHats}, 
{\ttfamily dTHatdpsiNs}, 
{\ttfamily dTHatdrHats}, or
{\ttfamily dTHatdrNs}
will be used.

\myhrule

\param{Zs}
{1D array of reals}
{1.0}
{Always}
{Charges of each species, in units of the proton charge $e$}

\myhrule

\param{mHats}
{1D array of reals}
{1.0}
{Always}
{Masses of each species, in units of the reference mass $\bar{m}$}

\myhrule

\param{nHats}
{1D array of reals}
{1.0}
{Whenever {\ttfamily RHSMode} == 1}
{Densities of each species, in units of the reference density $\bar{n}$}

\myhrule

\param{THats}
{1D array of reals}
{1.0}
{Whenever {\ttfamily RHSMode} == 1}
{Temperatures of each species, in units of the reference temperature $\bar{T}$}

\myhrule

\param{dnHatdpsiHats}
{1D array of reals}
{0.0}
{Whenever {\ttfamily RHSMode} == 1 and {\ttfamily inputRadialCoordinateForGradients} == 0}
{Radial density gradients of each species, with respect to the radial coordinate $\hat{\psi}$, normalized by the reference density $\bar{n}$.}

\myhrule

\param{dTHatdpsiHats}
{1D array of reals}
{0.0}
{Whenever {\ttfamily RHSMode} == 1 and {\ttfamily inputRadialCoordinateForGradients} == 0}
{Radial temperature gradients of each species, with respect to the radial coordinate $\hat{\psi}$, normalized by the reference temperature $\bar{T}$.}

\myhrule

\param{dnHatdpsiNs}
{1D array of reals}
{0.0}
{Whenever {\ttfamily RHSMode} == 1 and {\ttfamily inputRadialCoordinateForGradients} == 1}
{Radial density gradients of each species, with respect to the radial coordinate $\psi_N$, normalized by the reference density $\bar{n}$.}

\myhrule

\param{dTHatdpsiNs}
{1D array of reals}
{0.0}
{Whenever {\ttfamily RHSMode} == 1 and {\ttfamily inputRadialCoordinateForGradients} == 1}
{Radial temperature gradients of each species, with respect to the radial coordinate $\psi_N$, normalized by the reference temperature $\bar{T}$.}

\myhrule

\param{dnHatdrHats}
{1D array of reals}
{0.0}
{Whenever {\ttfamily RHSMode} == 1 and {\ttfamily inputRadialCoordinateForGradients} == 2}
{Radial density gradients of each species, with respect to the radial coordinate $\hat{r}$, normalized by the reference density $\bar{n}$.}

\myhrule

\param{dTHatdrHats}
{1D array of reals}
{0.0}
{Whenever {\ttfamily RHSMode} == 1 and {\ttfamily inputRadialCoordinateForGradients} == 2}
{Radial temperature gradients of each species, with respect to the radial coordinate $\hat{r}$, normalized by the reference temperature $\bar{T}$.}

\myhrule

\param{dnHatdrNs}
{1D array of reals}
{0.0}
{Whenever {\ttfamily RHSMode} == 1 and {\ttfamily inputRadialCoordinateForGradients} == 3}
{Radial density gradients of each species, with respect to the radial coordinate $r_N$, normalized by the reference density $\bar{n}$.}

\myhrule

\param{dTHatrNs}
{1D array of reals}
{0.0}
{Whenever {\ttfamily RHSMode} == 1 and {\ttfamily inputRadialCoordinateForGradients} == 3}
{Radial temperature gradients of each species, with respect to the radial coordinate $r_N$, normalized by the reference temperature $\bar{T}$.}










\section{The {\ttfamily physicsParameters}~namelist}

The parameters in this namelist determine which terms are included or excluded in the kinetic equation.
You will want to be aware of most of these parameters.

\myhrule

\param{Delta}
{real}
{4.5694e-3}
{Always}
{Roughly speaking, {\ttfamily Delta} is $\rho_*$ at the reference parameters. The precise definition is
\begin{eqnarray}
\mbox{\ttfamily Delta} 
&=& \frac{c \bar{m} \bar{v}}{e \bar{B} \bar{R}} \;\;\; \mbox{(Gaussian units)} \\
&=&  \frac{\bar{m} \bar{v}}{e \bar{B} \bar{R}} \;\;\; \mbox{(SI units)}, \nonumber
\end{eqnarray}
where $c$ is the speed of light,
$e$ is the proton mass,
and quantities with a bar are the normalization reference parameters discussed in section \ref{sec:normalizations}.
The default value {\ttfamily Delta} = 4.5694e-3 corresponds to $\bar{B}$ = 1 Tesla, $\bar{R}$ = 1 meter,
$\bar{m}$ = proton mass, and $\bar{T}$ = 1 keV.}

\myhrule

\param{alpha}
{real}
{1.0}
{Whenever {\ttfamily RHSMode} == 1 and $E_r$ is nonzero.}
{{\ttfamily alpha} $= e \bar{\Phi}/ \bar{T}$ (both Gaussian and SI units) where $e$ is the proton mass,
and $\bar{\Phi}$ and $\bar{T}$ are the normalization reference parameters discussed in section \ref{sec:normalizations}.
The default value {\ttfamily alpha} = 1.0 corresponds to $\bar{T}$ = 1 keV and $\bar{\Phi}$ = 1 kV.
The default value {\ttfamily alpha} = 1.0 also corresponds to $\bar{T}$ = 1 eV and $\bar{\Phi}$ = 1 V.
}

\myhrule

\param{EParallelHat}
{real}
{0.0}
{Whenever {\ttfamily RHSMode} == 1}
{Inductive parallel electric field:
\begin{equation}
\mbox{\ttfamily EParallelHat} = \left< \vect{E}\cdot\vect{B}\right> \frac{\bar{R}}{\bar{\Phi}\bar{B}}
\end{equation}
(in both Gaussian and SI units) where $\left< \ldots \right>$ denotes a flux surface average,
$\vect{E}$ and $\vect{B}$ are the electric and magnetic field vectors, and 
quantities with a bar are the normalization reference parameters discussed in section \ref{sec:normalizations}.}

\myhrule

\param{dPhiHatdpsiHat}
{real}
{0.0}
{Whenever {\ttfamily RHSMode} == 1 and {\ttfamily inputRadialCoordinateForGradients} == 0}
{The derivative of the electrostatic potential with respect to the radial coordinate $\hat{\psi}$,
i.e. the radial electric field up to a constant.
Notice that exactly 1 of the 4 variables {\ttfamily dPhiHatdpsiHat}, {\ttfamily dPhiHatdpsiN}, {\ttfamily dPhiHatdrHat}, and {\ttfamily dPhiHatdrN}
will be used, depending on\\
{\ttfamily inputRadialCoordinateForGradients}.
}

\myhrule

\param{dPhiHatdpsiN}
{real}
{0.0}
{Whenever {\ttfamily RHSMode} == 1 and {\ttfamily inputRadialCoordinateForGradients} == 1}
{The derivative of the electrostatic potential with respect to the radial coordinate $\psi_N$,
i.e. the radial electric field up to a constant.
Notice that exactly 1 of the 4 variables {\ttfamily dPhiHatdpsiHat}, {\ttfamily dPhiHatdpsiN}, {\ttfamily dPhiHatdrHat}, and {\ttfamily dPhiHatdrN}
will be used, depending on\\
{\ttfamily inputRadialCoordinateForGradients}.
}

\myhrule

\param{dPhiHatdrHat}
{real}
{0.0}
{Whenever {\ttfamily RHSMode} == 1 and {\ttfamily inputRadialCoordinateForGradients} == 2}
{The derivative of the electrostatic potential with respect to the radial coordinate $\hat{r}$,
i.e. the radial electric field up to a constant.
Notice that exactly 1 of the 4 variables {\ttfamily dPhiHatdpsiHat}, {\ttfamily dPhiHatdpsiN}, {\ttfamily dPhiHatdrHat}, and {\ttfamily dPhiHatdrN}
will be used, depending on\\
{\ttfamily inputRadialCoordinateForGradients}.
}

\myhrule

\param{dPhiHatdrN}
{real}
{0.0}
{Whenever {\ttfamily RHSMode} == 1 and {\ttfamily inputRadialCoordinateForGradients} == 3}
{The derivative of the electrostatic potential with respect to the radial coordinate $r_N$,
i.e. the radial electric field up to a constant.
Notice that exactly 1 of the 4 variables {\ttfamily dPhiHatdpsiHat}, {\ttfamily dPhiHatdpsiN}, {\ttfamily dPhiHatdrHat}, and {\ttfamily dPhiHatdrN}
will be used, depending on\\
{\ttfamily inputRadialCoordinateForGradients}.
}

\myhrule

\param{nu\_n}
{real}
{8.330e-3}
{Whenever {\ttfamily RHSMode} == 1}
{Dimensionless collisionality at the reference parameters:
\begin{equation}
\mbox{\ttfamily nu\_n} = \bar{\nu} \frac{\bar{R}}{\bar{v}},
\end{equation}
where $\bar{R}$ and $\bar{v}$ are the normalization reference parameters discussed in section \ref{sec:normalizations},
and $\bar{\nu}$ is the dimensional collision frequency at the reference parameters. This frequency is defined as
\begin{eqnarray}
\bar{\nu}
&=& \frac{4\sqrt{2\pi} \bar{n} e^4 \ln\Lambda}{3 (4\pi\epsilon_0)^2 \sqrt{\bar{m}} \bar{T}^{3/2}} \;\;\; \mbox{(SI units}) \\
&=& \frac{4\sqrt{2\pi} \bar{n} e^4 \ln\Lambda}{3 \sqrt{\bar{m}} \bar{T}^{3/2}} \;\;\; \mbox{(Gaussian units}) \nonumber
\end{eqnarray}
where $e$ is the proton charge, $\bar{n}$, $\bar{m}$, and $\bar{T}$ are the normalization reference parameters discussed in section \ref{sec:normalizations},
and $\ln\Lambda$ is the Coulomb logarithm.
The default value {\ttfamily nu\_n} = 8.330e-3 corresponds to $\bar{R}$ = 1 meter,
$\bar{m}$ = proton mass, $\bar{n}$ = $10^{20}$ m$^{-3}$, $\bar{T}$ = 1 keV, and $\ln\Lambda = 17$.
}

\myhrule

\param{nuPrime}
{real}
{1.0}
{Only when {\ttfamily RHSMode} == 3.}
{Dimensionless collisionality used in place of {\ttfamily nHats}, {\ttfamily THats}, {\ttfamily mHats}, {\ttfamily Zs}, and {\ttfamily nu\_n} for computing monoenergetic transport coefficients.
See section \ref{sec:monoenergetic} for more details.}

\myhrule

\param{EStar}
{real}
{0.0}
{Only when {\ttfamily RHSMode} == 3.}
{Normalized radial electric field used in place of {\ttfamily dPhiHatdXXX} for computing monoenergetic transport coefficients.
See section \ref{sec:monoenergetic} for more details.}


\myhrule

\param{collisionOperator}
{integer}
{0}
{Always}
{Which collision operator to use:\\

{\setlength{\parindent}{0cm}

{\ttfamily collisionOperator} = 0: Full linearized Fokker-Planck operator.\\

{\ttfamily collisionOperator} = 1: Pitch-angle scattering operator (with no momentum-conserving field term).
}}

\myhrule

\param{constraintScheme}
{integer}
{-1}
{Always}
{Controls a small number of extra rows and columns of the system matrix
which (1) eliminate the null space of the matrix, and (2) ensure that a steady-state solution
to the kinetic equation exists even when phase-space volume and/or energy are not conserved.
These issues are detailed in section III of Ref \cite{sfincsPaper}.
\\

{\setlength{\parindent}{0cm}

{\ttfamily constraintScheme} = -1: Automatic. If {\ttfamily collisionOperator}==0 then {\ttfamily constraintScheme} will be set to 1, 
otherwise {\ttfamily constraintScheme} will be set to 2.\\

{\ttfamily constraintScheme} = 0: No constraints.\\

{\ttfamily constraintScheme} = 1: 2 constraints per species: $\left<n_1\right>=0$ and $\left<p_1\right>=0$.\\

{\ttfamily constraintScheme} = 2: \Nx~constraints per species: $\left< f(L=0)\right>=0$ at each $x$.\\

You should set {\ttfamily constraintScheme} to -1 unless you know what you are doing.
}}

\myhrule

\param{includeXDotTerm}
{Boolean}
{\true}
{Whenever {\ttfamily RHSMode} $<3$ and the radial electric field is nonzero.}
{Whether or not to include the term in the kinetic equation corresponding to a change
in speed proportional to the radial electric field.  This term is given by $\dot{x}$ in
equation (17) of \cite{sfincsPaper}.}

\myhrule

\param{includeElectricFieldTermInXiDot}
{Boolean}
{\true}
{Whenever {\ttfamily RHSMode} $<3$ and the radial electric field is nonzero.}
{Whether or not to include the term in the kinetic equation corresponding to a change
in pitch angle $\xi$ proportional to the radial electric field.  This term is given by the last line of
equation (17) of \cite{sfincsPaper}.}

\myhrule

\param{useDKESExBDrift}
{Boolean}
{\false}
{Whenever {\ttfamily RHSMode} $<3$ and the radial electric field is nonzero.}
{If true, the $\vect{E}\times\vect{B}$ drift term 
multiplying $\partial f/\partial \theta$ and  $\partial f/\partial \zeta$
is taken to be $\vect{E}\times\vect{B}\cdot \nabla (\theta \;\mbox{or} \;\zeta) / \left< B^2 \right>$
instead of 
 $\vect{E}\times\vect{B}\cdot \nabla (\theta \;\mbox{or} \;\zeta) /B^2$.
}

\myhrule

\param{include\_fDivVE\_term}
{Boolean}
{\false}
{Never}
{Obsolete}

\myhrule

\param{includePhi1}
{Boolean}
{\false}
{Whenever {\ttfamily RHSMode} == 1.}
{If false, no terms involving $\Phi_1 = \Phi-\left<\Phi\right>$ are included in the kinetic equation,
and the quasineutrality equation is not solved.  If true, then terms involving $\Phi_1$ are included
in the kinetic equation, and the quasineutrality equation is solved at each point on the flux surface.
In this latter case, many more quantities are computed and saved in the output file, such as radial fluxes
associated with the radial $\vect{E}\times\vect{B}$ drift.}

\myhrule

\param{nonlinear}
{Boolean}
{\false}
{Whenever {\ttfamily RHSMode} == 1.}
{If true, certain terms that are nonlinear in the unknowns will be included in the kinetic equation.
Newton's method will be used to solve the nonlinear system, meaning that the usual linear solve in \sfincs~must
be iterated several times.
Running with {\ttfamily nonlinear}=\true requires {\ttfamily includePhi1}=\true
}

\myhrule

\param{includeTemperatureEquilibrationTerm}
{Boolean}
{\false}
{Whenever {\ttfamily RHSMode} == 1.}
{When true, the term $C_{ab}[ f_{Ma}, f_{Mb}]$ is included in the kinetic equation,
i.e. collisions between the leading-order Maxwellians of different species.  This term is nonzero
when the temperature is not the same for all species.  The resulting contribution to the non-Maxwellian
distribution function is isotropic and so does not directly give any parallel or radial transport.}

\myhrule

\param{magneticDriftScheme}
{integer}
{0}
{Whenever {\ttfamily RHSMode} == 1.}
{This variable controls the poloidal and magnetic drifts, and does not affect the radial magnetic drift.
THIS FUNCTIONALITY IS WORK IN PROGRESS, AND RESULTS ARE BUGGY WHEN THIS VARIABLE IS NONZERO.\\

{\ttfamily magneticDriftScheme} = 0:  No poloidal or toroidal magnetic drift.\\

{\ttfamily magneticDriftScheme} = 1: Use the grad-B and curvature drift, plus the parallel velocity correction $v_{\perp}^2/(2 \Omega_c) \vect{b} \vect{b} \cdot \nabla \times \vect{b}$.\\

{\ttfamily magneticDriftScheme} = 2:   Use the magnetic drift $v_m = (v_{||} / \Omega_c) \nabla (v_{||} \vect{b})$.
}








\section{The {\ttfamily resolutionParameters}~namelist}

In this namelist, there are 4 parameters you definitely need to be aware of and adjust: \Ntheta, \Nzeta, \Nxi, and \Nx.  
See chapter \ref{ch:resolution} for details. You may or may not
need to adjust {\ttfamily solverTolerance}.  The other parameters in this namelist almost never need to be adjusted.

\myhrule

\param{Ntheta}
{integer}
{15}
{Always}
{Number of grid points in the poloidal angle. 
This parameter should be odd; see {\ttfamily forceOddNthetaAndNzeta} in this namelist.
Memory and time requirements DO depend strongly on this parameter.
For stellarator calculations, this parameter can usually be in the range 15-25.
For tokamak calculations at low collisionality, the value of this parameter may need to be higher.}


\myhrule

\param{Nzeta}
{integer}
{15}
{Always}
{Number of grid points in the toroidal angle (per identical segment of the stellarator.)  
This parameter should be odd; see {\ttfamily forceOddNthetaAndNzeta} in this namelist.
Memory and time requirements DO depend strongly on this parameter.
Set this parameter to 1 for a tokamak calculation.
For stellarator calculations, the value of this parameter required for convergence depends strongly on
the collisionality. At high collisionality, this parameter can be several 10s, depending
on the complexity of $B(\theta,\zeta)$. At low collisionality, this parameter may need to be many 10s or
even $>$ 100 for convergence.
}

\myhrule

\param{Nxi}
{integer}
{16}
{Always}
{Number of Legendre polynomials used to represent the pitch-angle dependence of the distribution function.
Memory and time requirements DO depend strongly on this parameter.
The value of this parameter required for convergence depends strongly on
the collisionality. At high collisionality, this parameter can be as low
as 5. At low collisionality, this parameter may need to be many 10s or
even $>$ 100 for convergence.}

\myhrule

\param{Nx}
{integer}
{5}
{Always}
{Number of grid points in energy used to represent the distribution function.
Memory and time requirements DO depend strongly on this parameter.
This parameter almost always needs to be at least 5.
Usually a value in the range 5-8 is plenty for convergence, though in exceptional circumstances
you may need to go up to 10-15.
}

\myhrule

\param{solverTolerance}
{real}
{1e-6}
{Whenever {\ttfamily useIterativeLinearSolver} == \true}
{
Tolerance used to define convergence of the Krylov solver.
This parameter does not affect memory requirements but it does affect the
time required for solution somewhat.
Occasionally you may want to ease this tolerance to 1e-5 so fewer iterations of the Krylov solver are needed.
}

\myhrule

\param{NL}
{integer}
{4}
{Whenever {\ttfamily collisionOperator} == 0.}
{Number of Legendre polynomials used to represent the Rosenbluth
potentials. This number can basically always be 4,
since results barely change when {\ttfamily NL} is increased above this value.
Memory and time requirements do NOT depend strongly on this parameter.
}

\myhrule

\param{NxPotentialsPerVth}
{real}
{40.0}
{Only when {\ttfamily collisionOperator} == 0 and {\ttfamily xGridScheme} $<$ 5.
Since {\ttfamily xGridScheme} = 5 is recommended, this parameter is basically obsolete.}
{
Number of grid points in energy used to represent the Rosenbluth potentials
for the original implementation of the Fokker-Planck operator described in \cite{speedGrids}.
Memory and time requirements do NOT depend strongly on this parameter.}

\myhrule

\param{xMax}
{real}
{5.0}
{Only when {\ttfamily collisionOperator} == 0 and {\ttfamily xGridScheme} $<$ 5.
Since {\ttfamily xGridScheme} = 5 is recommended, this parameter is basically obsolete.}
{
Maximum normalized speed for the Rosenbluth potential grid
for the original implementation of the Fokker-Planck operator described in \cite{speedGrids}.
Memory and time requirements do NOT depend strongly on this parameter.}

\myhrule

\param{forceOddNthetaAndNzeta}
{Boolean}
{\true}
{Always}
{
If true, 1 is added to \Ntheta~any time a run is attempted with even \Ntheta,
and 1 is added to \Nzeta~any time a run is attempted with even \Nzeta.
When false, the even and odd grid points are effectively decoupled so results are unstable.
This parameter should be true unless you know what you are doing.
}














\section{The {\ttfamily otherNumericalParameters}~namelist}

The parameters in this namelist are advanced, and the default values are best for routine use of the code.

\myhrule

\param{thetaDerivativeScheme}
{integer}
{2}
{Always}
{Discretization scheme for the poloidal angle coordinate theta.\\

{\ttfamily thetaDerivativeScheme} = 0: Fourier spectral collocation.  The differentiation matrix in theta is dense.\\

{\ttfamily thetaDerivativeScheme} = 1: Finite differences with a 3 point stencil.  (The differentiation matrix in theta is tridiagonal, aside from the corners.)\\

{\ttfamily thetaDerivativeScheme} = 2: Finite differences with a 5 point stencil.  (The differentiation matrix in theta is pendadiagonal, aside from the corners.).\\

The best value for this parameter is usually 2.}


\myhrule

\param{zetaDerivativeScheme}
{integer}
{2}
{Always}
{Discretization scheme for the toroidal angle coordinate zeta.\\

{\ttfamily zetaDerivativeScheme} = 0: Fourier spectral collocation.  The differentiation matrix in zeta is dense.\\

{\ttfamily zetaDerivativeScheme} = 1: Finite differences with a 3 point stencil.  (The differentiation matrix in zeta is tridiagonal, aside from the corners.)\\

{\ttfamily zetaDerivativeScheme} = 2: Finite differences with a 5 point stencil.  (The differentiation matrix in zeta is pendadiagonal, aside from the corners.).\\

The best value for this parameter is usually 2.}

\myhrule

\param{xGridScheme}
{integer}
{5}
{Whenever {\ttfamily RHSMode} is 1 or 2.}
{Discretization scheme for the speed coordinate $x$.\\

{\ttfamily xGridScheme} = 1: New orthogonal polynomials with no point at $x=0$. Original treatment of Rosenbluth potentials.\\

{\ttfamily xGridScheme} = 2: New orthogonal polynomials with a point at $x=0$. Original treatment of Rosenbluth potentials.\\

{\ttfamily xGridScheme} = 3: Uniform finite differences on [0, {\ttfamily xMax}], forcing $f=0$ at {\ttfamily xMax}. 2-point stencil for interpolating to other grids.\\

{\ttfamily xGridScheme} = 4: Uniform finite differences on [0, {\ttfamily xMax}], forcing $f=0$ at {\ttfamily xMax}. 4-point stencil for interpolating to other grids.\\

{\ttfamily xGridScheme} = 5: New orthogonal polynomials with no point at $x=0$. New treatment of Rosenbluth potentials.\\

{\ttfamily xGridScheme} = 6: New orthogonal polynomials with a point at $x=0$. New treatment of Rosenbluth potentials.\\

The recommended value for this parameter is 5.  
When {\ttfamily xGridScheme} = 5 or 6, then the following quantities do not matter: 
{\ttfamily NxPotentialsPerVth}, {\ttfamily xMax}, and {\ttfamily xPotentialsGridScheme}.}

\myhrule

\param{xGrid\_k}
{integer}
{0}
{Whenever {\ttfamily RHSMode} is 1 or 2 and {\ttfamily xGridScheme} = 1, 2, 5, or 6.}
{For {\ttfamily xGridScheme} = 1, 2, 5, or 6, the distribution function will be represented in terms of polynomials $P_n(x)$
that are orthogonal under the weight $\int_0^\infty dx\; x^k \exp(-x^2) P_n(x) P_m(x)  \propto  \delta_{n,m}$
where $k$ is an exponent set by the parameter {\ttfamily xGrid\_k} here.
A good value to use is 0, 1, or 2.}

\myhrule

\param{xPotentialsGridScheme}
{integer}
{2}
{Whenever {\ttfamily RHSMode} is 1 or 2 and {\ttfamily xGridScheme} is $<$5.
Since the recommended setting for {\ttfamily xGridScheme} is 5, this parameter is rarely relevant.}
{When an explicit grid is used for the Rosenbluth potentials, which grid and interpolation scheme to use.\\

{\ttfamily xPotentialsGridScheme} = 1: Uniform grid. 5-point stencil for derivatives. 2-point stencil for interpolating to other grids.\\

{\ttfamily xPotentialsGridScheme} = 2: Uniform grid. 5-point stencil for derivatives. 4-point stencil for interpolating to other grids.\\

{\ttfamily xPotentialsGridScheme} = 3: Use same grid as for distribution function, so no interpolation needed for the self-collision operator.
  You must set {\ttfamily xGridScheme} = 3 or 4 to use this setting.  Use 2-point stencil for interpolating to other species' grids.\\

{\ttfamily xPotentialsGridScheme} = 4: Same as option 3, except use a 4-point stencil for interpolating to other species' grids.\\

The recommended setting is {\ttfamily xPotentialsGridScheme} = 2.}

\myhrule

\param{useIterativeLinearSolver}
{Boolean}
{\true}
{Always}
{If false, a sparse direct solver
will be used. The direct solver is faster for small (i.e. low-resolution) problems and always yields a solution
(as long as there is sufficient memory).
For large (high resolution) problems, the iterative solver will usually be faster and will use much
 less memory, but it may not always converge.
}

\myhrule

\param{whichParallelSolverToFactorPreconditioner}
{integer}
{1}
{Always}
{Which software package is used to $LU$-factorize the preconditioner matrix.\\

{\ttfamily whichParallelSolverToFactorPreconditioner} = 1: Use \mumps~if it is available, otherwise use \superludist.\\

{\ttfamily whichParallelSolverToFactorPreconditioner} = 2: Force use of \superludist~ even if \mumps~is available.
}

\myhrule

\param{PESCPreallocationStrategy}
{integer}
{1}
{Always}
{This setting changes the estimated number of nonzeros (nnz) used for allocating memory for the system matrix and preconditioner.\\

{\ttfamily PESCPreallocationStrategy} = 0: Old method with high estimated nnz. This method involves relatively simpler code but uses WAY more memory than necessary.\\

{\ttfamily PESCPreallocationStrategy} = 1: New method with lower, more precise estimated nnz.  This method should use much less memory.\\

Use {\ttfamily PETSCPreallocationStrategy} = 1 unless you know what you are doing.}









\section{The {\ttfamily preconditionerOptions}~namelist}

This namelist controls how elements are removed from the ``real'' matrix in order to obtain
the preconditioner matrix. The default values are usually best, but if you find that there are more than 100 iterations
of GMRES/KSP, it may be worth adjusting these settings.  As long as KSP converges, these parameters
should have no impact (to several digits) on the physical outputs such as parallel flows and radial fluxes.
Therefore, do not worry about (for example) ``dropping coupling between species'' in the first
parameter below, since full inter-species coupling will be retained in the real equations that are being solved.

\myhrule

\param{preconditioner\_species}
{integer}
{1}
{Whenever {\ttfamily useIterativeLinearSolver} = \true~and there are 2 or more species.}
{\\
{\ttfamily preconditioner\_species} = 0: Keep all coupling between species.\\

{\ttfamily preconditioner\_species} = 1: Drop all coupling between species.\\

The default value of 1 is recommended, except perhaps at high collisionality where 0 may be preferable.}

\myhrule

\param{preconditioner\_x}
{integer}
{1}
{Whenever {\ttfamily useIterativeLinearSolver} = \true~and {\ttfamily RHSMode} = 1 or 2.}
{\\
{\ttfamily preconditioner\_x} = 0: Keep full $x$ coupling.\\

{\ttfamily preconditioner\_x} = 1: Drop everything off-diagonal in $x$.\\

{\ttfamily preconditioner\_x} = 2: Keep only upper-triangular part in $x$.\\

{\ttfamily preconditioner\_x} = 3: Keep only the tridiagonal terms in $x$.\\

{\ttfamily preconditioner\_x} = 4: Keep only the diagonal and superdiagonal in $x$.\\

The default value of 1 is strongly recommended, except perhaps at high collisionality where 0 may be preferable.}

\myhrule

\param{preconditioner\_x\_min\_L}
{integer}
{0}
{Whenever {\ttfamily useIterativeLinearSolver} = \true~and {\ttfamily RHSMode} = 1 or 2 and {\ttfamily preconditioner\_x} $>$ 0.}
{The $x$ structure of the matrix will only be simplified for Legendre index $L$ is $\ge$ this value.
Set {\ttfamily preconditioner\_x\_min\_L} = 0 to simplify the matrix for every $L$.
Recommended values are 0, 1, or 2.}

\myhrule

\param{preconditioner\_theta}
{integer}
{0}
{Whenever {\ttfamily useIterativeLinearSolver} = \true}
{\\
{\ttfamily preconditioner\_theta} = 0: Keep full $\theta$ coupling.\\

{\ttfamily preconditioner\_theta} = 1: Use a 3-point finite difference stencil for $d/d\theta$.\\

{\ttfamily preconditioner\_theta} = 2: Drop all $\theta$ coupling.\\

{\ttfamily preconditioner\_theta} = 3: Replace $d/d\theta$ with the identity matrix.\\

The default value of 0 is strongly recommended.}

\myhrule

\param{preconditioner\_theta\_min\_L}
{integer}
{0}
{Whenever {\ttfamily useIterativeLinearSolver} = \true~and {\ttfamily preconditioner\_theta} $>$ 0.}
{The $\theta$ structure of the matrix will only be simplified for Legendre index $L$ is $\ge$ this value.
Set {\ttfamily preconditioner\_theta\_min\_L} = 0 to simplify the matrix for every $L$.
}

\myhrule

\param{preconditioner\_zeta}
{integer}
{0}
{Whenever {\ttfamily useIterativeLinearSolver} = \true}
{\\
{\ttfamily preconditioner\_zeta} = 0: Keep full $\zeta$ coupling.\\

{\ttfamily preconditioner\_zeta} = 1: Use a 3-point finite difference stencil for $d/d\zeta$.\\

{\ttfamily preconditioner\_zeta} = 2: Drop all $\zeta$ coupling.\\

{\ttfamily preconditioner\_zeta} = 3: Replace $d/d\zeta$ with the identity matrix.\\

The default value of 0 is strongly recommended.}

\myhrule

\param{preconditioner\_zeta\_min\_L}
{integer}
{0}
{Whenever {\ttfamily useIterativeLinearSolver} = \true~and {\ttfamily preconditioner\_zeta} $>$ 0.}
{The $\zeta$ structure of the matrix will only be simplified for Legendre index $L$ is $\ge$ this value.
Set {\ttfamily preconditioner\_zeta\_min\_L} = 0 to simplify the matrix for every $L$.
}

\myhrule

\param{preconditioner\_xi}
{integer}
{1}
{Whenever {\ttfamily useIterativeLinearSolver} = \true}
{\\
{\ttfamily preconditioner\_xi} = 0: Keep full $\xi$ coupling.\\

{\ttfamily preconditioner\_xi} = 1: Drop terms that are $\pm 2$ rows from the diagonal in $\xi$, 
      so the preconditioner matrix becomes tridiagonal in $\xi$.
(Normally the preconditioner matrix is pentadiagonal in $\xi$.)\\

Either a setting of 0 or 1 can be good for this parameter.}

\myhrule

\param{reusePreconditioner}
{Boolean}
{\true}
{Only when {\ttfamily nonlinear} = \true}
{If true, the nonlinear term will not be included in the preconditioner matrix, meaning the preconditioner matrix is the same
at every iteration, and so the preconditioner matrix only needs to be $LU$-factorized once.  If false, the preconditioner matrix
for the Jacobian will be different at each iteration of the Newton solve, so the preconditioner needs to be $LU$-factorized at
each iteration.  The nonlinear term also introduces a lot of nonzeros into the preconditioner matrix, so setting
{\ttfamily reusePreconditioner =}\true~not only dramatically reduces the time required for a nonlinear calculation, but also the memory required.}










\section{The {\ttfamily export\_f}~namelist}

This namelist controls whether and how the distribution function is saved in {\ttfamily sfincsOutput.h5}.
For each of the 4 coordinates $(\theta, \zeta, x, \xi)$, the distribution function can be given with the same discretization
used for solving the kinetic equation, or you can interpolate to a different grid/discretization.
For all available settings, the distribution function will be reported on a tensor product grid in the 4 coordinates.

\myhrule

\param{export\_full\_f}
{Boolean}
{\false}
{Always}
{Whether or not to save the full distribution function (the sum of the leading-order Maxwellian and the departure from it)
in the output file.}

\myhrule

\param{export\_delta\_f}
{Boolean}
{\false}
{Always}
{Whether or not to save the departure from a Maxwellian distribution function in the output file.}

\myhrule

\param{export\_f\_theta\_option}
{integer}
{2}
{Whenever {\ttfamily export\_full\_f} or {\ttfamily export\_delta\_f} is \true}
{Controls which grid in $\theta$ is used for exporting the distribution function.\\

{\ttfamily export\_f\_theta\_option} = 0: Report the distribution function on the original $\theta$ grid (with \Ntheta~points) used for solving the kinetic equation.\\

{\ttfamily export\_f\_theta\_option} = 1: Interpolate to a different grid, specified by {\ttfamily export\_f\_theta}. Linear interpolation will be used.
No sorting of the requested values is performed.\\

{\ttfamily export\_f\_theta\_option} = 2: Do not interpolate. Use the
values of the $\theta$ grid that are closest to the values requested
in {\ttfamily export\_f\_theta}.  Values of $\theta$ will be in
increasing order. If multiple requested values are close to the same
grid point, the number of points returned will be less than the number
of points requested. \\

For all of these options, you can see {\ttfamily export\_f\_theta} in {\ttfamily sfincsOutput.h5} for the actual grid used in the end.
}

\myhrule

\param{export\_f\_zeta\_option}
{integer}
{2}
{Whenever {\ttfamily export\_full\_f} or {\ttfamily export\_delta\_f} is \true}
{Controls which grid in $\zeta$ is used for exporting the distribution function.\\

{\ttfamily export\_f\_zeta\_option} = 0: Report the distribution function on the original $\zeta$ grid (with \Nzeta~points) used for solving the kinetic equation.\\

{\ttfamily export\_f\_zeta\_option} = 1: Interpolate to a different grid, specified by {\ttfamily export\_f\_zeta}. Linear interpolation will be used.
No sorting of the requested values is performed.\\

{\ttfamily export\_f\_zeta\_option} = 2: Do not interpolate. Use the
values of the $\zeta$ grid that are closest to the values requested
in {\ttfamily export\_f\_zeta}.  Values of $\zeta$ will be in
increasing order. If multiple requested values are close to the same
grid point, the number of points returned will be less than the number
of points requested. \\

For all of these options, you can see {\ttfamily export\_f\_zeta} in {\ttfamily sfincsOutput.h5} for the actual grid used in the end.
}

\myhrule

\param{export\_f\_theta}
{1D array of reals}
{0.0}
{Whenever {\ttfamily export\_full\_f} or {\ttfamily export\_delta\_f} is \true, and {\ttfamily export\_f\_theta\_option} $>0$.}
{Values of $\theta$ on which you want to save the distribution function.  modulo$(\ldots, 2\pi)$ will be applied.  See {\ttfamily export\_f\_theta\_option} for details}


\myhrule

\param{export\_f\_zeta}
{1D array of reals}
{0.0}
{Whenever {\ttfamily export\_full\_f} or {\ttfamily export\_delta\_f} is \true, and {\ttfamily export\_f\_zeta\_option} $>0$.}
{Values of $\zeta$ on which you want to save the distribution function.  modulo$(\ldots, 2\pi/\mbox{\ttfamily NPeriods})$ will be applied.  See {\ttfamily export\_f\_zeta\_option} for details}

\myhrule

\param{export\_f\_xi\_option}
{integer}
{1}
{Whenever {\ttfamily export\_full\_f} or {\ttfamily export\_delta\_f} is \true}
{Controls which discretization in $\xi$ is used for exporting the distribution function.\\

{\ttfamily export\_f\_xi\_option} = 0: Report the distribution function as amplitudes of \Nxi~Legendre polynomials, as used internally by \sfincs~for solving the kinetic equation.\\

{\ttfamily export\_f\_xi\_option} = 1: Report the distribution function on the values of $\xi$ specified by {\ttfamily export\_f\_xi}. No sorting of the requested values is performed.
}

\myhrule

\param{export\_f\_xi}
{1D array of reals}
{0.0}
{Whenever {\ttfamily export\_full\_f} or {\ttfamily export\_delta\_f} is \true, and {\ttfamily export\_f\_xi\_option} = 1.}
{Values of $\xi$ on which you want to save the distribution function.  Values must lie in the range $[-1,1]$.}

\myhrule

\param{export\_f\_x\_option}
{integer}
{0}
{Whenever {\ttfamily export\_full\_f} or {\ttfamily export\_delta\_f} is \true}
{Controls which grid in $x = v/\sqrt{2T/m}$ is used for exporting the distribution function.\\

{\ttfamily export\_f\_x\_option} = 0: Report the distribution function on the original $x$ grid (with \Nx~points) used for solving the kinetic equation.\\

{\ttfamily export\_f\_x\_option} = 1: Interpolate to a different grid, specified by {\ttfamily export\_f\_x}.  Polynomial spectral interpolation will be used.
  No sorting of the requested values is performed.\\

{\ttfamily export\_f\_x\_option} = 2: Do not interpolate. Use the values of the internal $x$ grid that are closest to the values requested in {\ttfamily export\_f\_x}.
  Values of $x$ will be in increasing order. If multiple requested values are close to the same grid point, the number of
  points returned will be less than the number of points requested.\\

For all of these options, you can see {\ttfamily export\_f\_x} in {\ttfamily sfincsOutput.h5} for the actual grid used in the end.
}

\myhrule

\param{export\_f\_x}
{1D array of reals}
{1.0}
{Whenever {\ttfamily export\_full\_f} or {\ttfamily export\_delta\_f} is \true, and {\ttfamily export\_f\_x\_option} $>0$.}
{Values of $x$ on which you want to save the distribution function.  Values must be $\ge 0$.}




















\section{Directives for \sfincsScan}
\label{sec:sfincsScanParams}

The parameters for \sfincsScan~ begin with the code {\ttfamily !ss} and so are not read by the fortran part of \sfincs.
These parameters matter only when \sfincsScan~is called and are all ignored when \sfincs~is executed directly.
These parameters can appear anywhere in the {\ttfamily input.namelist} file, in any namelist or outside of any namelist.
Note that \sfincsScan~ parameters do not have defaults, unlike fortran namelist parameters.\\


The resolution parameters each have 3 associated sfincsScan parameters
which are used for convergence scans ({\ttfamily scanType} = 1): {\ttfamily ...MinFactor}, {\ttfamily ...MaxFactor}, and {\ttfamily ...NumRuns}.
The first two of these set the range by which the associated resolution parameter is scaled
in a convergence scan. The {\ttfamily ...NumRuns} parameter sets the number of values tried in a
convergence scan. The code attempts to space the values evenly in a logarithmic
sense, as in Matlab's 'logspace' function. For example, the following settings\\
{\ttfamily
    Nxi = 20\\
!ss NxiMinFactor = 0.5\\
!ss NxiMaxFactor = 2.0\\
!ss NxiNumRuns = 3\\
}
would mean the values {\ttfamily Nxi} = 10, 20, and 40 would be tried in a convergence scan.
If you don't want to scan a variable in a convergence scan, set the associated
{\ttfamily ...NumRuns} parameter to 0, or do not include this parameter in the input file.
For each resolution parameter ({\ttfamily Ntheta}, {\ttfamily Nzeta}, {\ttfamily Nxi}, etc.), the value itself is read by
Fortran and so should not be preceded by {\ttfamily !ss}.  However the {\ttfamily ...MinFactor}, {\ttfamily ...MaxFactor}, and {\ttfamily ...NumRuns}
quantities are read by \sfincsScan~and so must be preceded by {\ttfamily !ss}

\myhrule

\ssparam{scanType}
{integer}
{Any time \sfincsScan~is called.}
{Which type of scan will be run when \sfincsScan~is called. \\

{\ttfamily scanType} = 1:  Convergence scan. (Scan the parameters in the resolutionParameters namelist.)\\

{\ttfamily scanType} = 2:  Scan of $E_r$.\\

{\ttfamily scanType} = 3:  Scan any one input parameter that takes a numeric value.\\

{\ttfamily scanType} = 4:  Scan radius, taking the density and temperature profiles from the {\ttfamily profiles} file.
In this type of scan, the same radial electric field is used at every radius.
See \path{sfincs/fortran/utils/profiles.XXX} for examples.\\

{\ttfamily scanType} = 5:  Scan radius, and at each radius, scan $E_r$.  Density and temperature profiles are again
 taken from the {\ttfamily profiles} file; see \path{sfincs/fortran/utils/profiles.XXX} for examples.\\

{\ttfamily scanType} = 21: Read in a list of requested runs from a file {\ttfamily runspec.dat}. See \path{sfincs/fortran/utils/sfincsScan_21} for an example file.
}









\section{\PETSc~commands}

Command-line flags can be used to modify the behavior of any \PETSc~application, including \sfincs.
There are hundreds of \PETSc~options, and a list can be obtained by running with the command-line flag
{\ttfamily -help}. Here we list some of the more useful options.\\

\PETScParam{-help}
{Dumps a list of available command-line options to stdout.}

\myhrule

\PETScParam{-ksp\_view}
{Dumps detailed information to stdout related to the linear solver.}

\myhrule

\PETScParam{-ksp\_gmres\_restart {\normalfont \ttfamily$<$integer$>$}}
{After how many iterations will GMRES restart. Default is 2000. The convergence of GMRES slows every time a restart occurs, but restarts also free up memory.}

\myhrule

\PETScParam{-mat\_mumps\_icntl\_4 {\normalfont \ttfamily$<$integer$>$}}
{How much diagnostic information will be printed by \mumps. Default is 0. Set to 2 or 3 to print out useful diagnostic information about the memory required for factorizing the preconditioner.}


