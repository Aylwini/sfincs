\chapter{Specifying and running a computation}

\section{Normalizations}
\label{sec:normalizations}

{\setlength{\parindent}{0cm}
Dimensional quantities in \sfincs~are normalized to ``reference'' values that are denoted by a bar:\\
$\bar{B}$ = reference magnetic field, typically 1 Tesla.\\
$\bar{R}$ = reference length, typically 1 meter.\\
$\bar{n}$ = reference density, typically $10^{20}$ m$^{-3}$, $10^{19}$ m$^{-3}$, or something similar.\\
$\bar{m}$ = reference mass, typically either the mass of hydrogen or deuterium.\\
$\bar{T}$ = reference temperature in energy units, typically 1 eV or 1 keV.\\
$\bar{v} = \sqrt{2 \bar{T} / \bar{m}}$ = thermal speed at the reference temperature and mass\\
$\bar{\Phi}$ = reference electrostatic potential, typically 1 V or 1 kV.\\
}

You can choose any reference parameters you like, not just the values
suggested here. However, if you use a {\ttfamily vmec} or {\ttfamily .bc} magnetic equilibrium
by choosing \parlink{geometryScheme} = 5, 11, or 12, then you MUST use $\bar{B}$ = 1 Tesla and $\bar{R}$ = 1 meter.
The code ``knows'' about the reference values only through
the 3 combinations \parlink{Delta}, \parlink{alpha}, and \parlink{nu\_n}
in the {\ttfamily \hyperref[sec:physicsParameters]{physicsParameters}} namelist.

Normalized quantities are denoted by a ``hat''.  Taking the magnetic field as an example,
$\hat{B}=B/\bar{B}$, where $\hat{B}$ is called {\ttfamily BHat} in the fortran code and \HDF~output file.

\section{Radial coordinates}
\label{sec:radialCoordinates}

A variety of flux-surface label coordinates are used in other codes and in the literature.
One common choice (used in \vmec) is $\psi_N$, the toroidal flux normalized to its
value at the last closed flux surface.  Another common choice is an ``effective normalized minor radius''
$r_N$, defined by $r_N=\sqrt{\psi_N}$.  For gradients of density, temperature, and electrostatic potential (i.e. the radial
electric field), it is useful to use a dimensional local minor radius $r = r_N a$, where $a$ is some
measure of the plasma effective outer minor radius.  Finally, one could also use $\psi$ directly.
For maximum flexibility, \sfincs~permits any of these four radial coordinates to be used, and different radial
coordinates can be used in different aspects of a given computation.  Output quantities which depend
on the radial coordinate, such as radial fluxes, are often given with respect to all radial coordinates.
In \sfincs, the four radial coordinates are named as follows:\\

{\setlength{\parindent}{0cm}

{\ttfamily \hypertarget{psiHat}{psiHat}} = $\hat\psi$ is the toroidal flux (divided by $2\pi$), normalized by $\bar{B}\bar{R}^2$.\\

{\ttfamily \hypertarget{psiN}{psiN}} = $\psi_N$ is the toroidal flux normalized by its value at the last closed flux surface.\\

{\ttfamily \hypertarget{rHat}{rHat}} = $\hat{r}$ is defined as {\ttfamily aHat}$\sqrt{\mbox{\ttfamily psiN}}$, where {\ttfamily aHat} is an effective minor radius of the last closed flux surface normalized by $\bar{R}$.\\

{\ttfamily \hypertarget{rN}{rN}} = $r_N$ is defined as $\sqrt{\mbox{\ttfamily psiN}}$.\\

}

These four radial coordinates are identified by the numbers 0, 1, 2, and 3 respectively.

When setting up a run, you can make several independent choices for radial coordinates.  One parameter
you select is \parlink{inputRadialCoordinateForGradients} in the 
{\ttfamily \hyperref[sec:geometryParameters]{geometryParameters}} namelist.
This parameter controls which coordinate is used to specify the gradients. The possible values of \parlink{inputRadialCoordinateForGradients} are:\\

{\setlength{\parindent}{0cm}

0: Use derivatives with respect to {\ttfamily psiHat}: Density gradients are specified using \parlink{dnHatdpsiHats}, 
temperature gradients are specified using \parlink{dTHatdpsiHats},  a single $E_r$ is specified using
\parlink{dPhiHatdpsiHat}, and a range of $E_r$ for a scan is specified using \parlink{dPhiHatdpsiHatMin}-\parlink{dPhiHatdpsiHatMax}.
\\

1: Use derivatives with respect to {\ttfamily psiN}: Density gradients are specified using \parlink{dnHatdpsiNs}, 
temperature gradients are specified using \parlink{dTHatdpsiNs},  a single $E_r$ is specified using
\parlink{dPhiHatdpsiN}, and a range of $E_r$ for a scan is specified using \parlink{dPhiHatdpsiNMin}-\parlink{dPhiHatdpsiNMax}.
\\

2: Use derivatives with respect to {\ttfamily rHat}: Density gradients are specified using \parlink{dnHatdrHats}, 
temperature gradients are specified using \parlink{dTHatdrHats},  a single $E_r$ is specified using
\parlink{dPhiHatdrHat}, and a range of $E_r$ for a scan is specified using \parlink{dPhiHatdrHatMin}-\parlink{dPhiHatdrHatMax}.
\\

3: Use derivatives with respect to {\ttfamily rN}: Density gradients are specified using \parlink{dnHatdrNs}, 
temperature gradients are specified using \parlink{dTHatdrNs},  a single $E_r$ is specified using
\parlink{dPhiHatdrN}, and a range of $E_r$ for a scan is specified using \parlink{dPhiHatdrNMin}-\parlink{dPhiHatdrNMax}.
\\

}

Another choice involving radial coordinates is how to specify the flux surface for the computation.
This choice is made using the parameter \parlink{inputRadialCoordinate} in the {\ttfamily \hyperref[sec:geometryParameters]{geometryParameters}}
namelist, which is again an integer from 0 to 3, and this parameter need not be the same as \parlink{inputRadialCoordinateForGradients}.
An extra complication with specifying the flux surface is that the magnetic equilibrium file will contain data on a finite number of surfaces,
and you may wish to use one of these surfaces.  For this reason, the parameters for specifying the flux surface have {\ttfamily \_wish}
appended to the name. In other words, the allowed values for \parlink{inputRadialCoordinate} are:\\

{\setlength{\parindent}{0cm}

0: Specify the flux surface using \parlink{psiHat\_wish}.\\

1: Specify the flux surface using \parlink{psiN\_wish}.\\

2: Specify the flux surface using \parlink{rHat\_wish}.\\

3: Specify the flux surface using \parlink{rN\_wish}.\\

}

When using \parlink{geometryScheme} == 11 or 12, \sfincs~will always shift the ``wish'' value so it matches an available surface in the magnetic equilibrium file.
For \parlink{geometryScheme} == 5, the \parlink{VMECRadialOption} parameter lets you can choose whether to shift to the nearest surface in the magnetic equilibrium file,
or to interpolate the \vmec~data onto the exact value of radius you specify.

If you perform a radial scan, then there is a third choice you can make: which radial coordinate to use in the {\ttfamily profiles} file.
This choice is made with an integer 0, 1, 2, or 3 in the first non-comment line of the {\ttfamily profiles} file.
The radial coordinate used in the {\ttfamily profiles} file need not be the same as either
\parlink{inputRadialCoordinate} or \parlink{inputRadialCoordinateForGradients}.
Note however that the maximum and minimum radial electric field specified in the {\ttfamily profiles}
file must be defined as the derivative of the electrostatic potential with respect to the radial coordinate \parlink{inputRadialCoordinateForGradients}.

For more details about the behavior of \parlink{inputRadialCoordinate}, \parlink{inputRadialCoordinateForGradients}, and \parlink{VMECRadialOption},
see section \ref{sec:geometryParameters}.

\section{Trajectory models}

As discussed in \cite{sfincsPaper},
one of the capabilities of \sfincs~is to compare various models for the terms in the kinetic equation involving $E_r$.
These variations of the kinetic equation are called ``trajectory models'' in \cite{sfincsPaper}.
The relevant terms in the kinetic equation can be turned off and on by certain Boolean parameters in the {\ttfamily \hyperref[sec:physicsParameters]{physicsParameters}} namelist.
The models described in \cite{sfincsPaper} are selected as follows:\\

{\setlength{\parindent}{0cm}

\underline{Full trajectories:}\\
{\ttfamily 
\parlink{includeXDotTerm} = .true.\\
\parlink{includeElectricFieldTermInXiDot} = .true.\\
\parlink{useDKESExBDrift} = .false.\\
}

\underline{Partial trajectories:}\\
{\ttfamily
\parlink{includeXDotTerm} = .false.\\
\parlink{includeElectricFieldTermInXiDot} = .false.\\
\parlink{useDKESExBDrift} = .false.\\
}

\underline{DKES trajectories:}\\
{\ttfamily
\parlink{includeXDotTerm} = .false.\\
\parlink{includeElectricFieldTermInXiDot} = .false.\\
\parlink{useDKESExBDrift} = .true.\\
}
}

There is not a significant difference in computational cost between these models.


\section{Quasineutrality and variation of the electrostatic potential on the flux surface}

One choice you should consider in setting up a computation is whether or not
to include variation on the flux surface of the electrostatic potential, $\Phi_1(\theta,\zeta)$.
Such variation does occur to some degree in a real plasma, but it is 
neglected in analytical theory and in many codes such as \dkes.  It can be proved
that including $\Phi_1$ has no effect on the particle or heat fluxes, parallel flows, or bootstrap current
when $E_r=0$, but generally there can be some difference when $E_r \ne 0$.
(There is a subtlety in showing that the heat flux is the same with and without $\Phi_1$,
discussed in the notes 20150325-01 in \path{sfincs/doc/}.)

In \sfincs, you can choose whether or not to
include $\Phi_1$ using the paramter {\ttfamily \hyperlink{includePhi1}{includePhi1}} in the 
{\ttfamily \hyperref[sec:physicsParameters]{physicsParameters}}
namelist.  If and only if $\Phi_1$ is included, a quasineutrality equation is solved
at each point on the flux surface.  Due to these extra unknowns ($\Phi_1$) and extra equations
(quasineutrality), the system matrix is slightly larger when \parlink{includePhi1} is \true.
Specifically, the number of rows and columns are each increased by \Ntheta$\times$\Nzeta$+1$.  This increase is miniscule compared
to the number of rows and columns associated with the kinetic equation, which depends not only on real space
but also on velocity space and species.  Thus, there is very little extra computational cost associated
with \parlink{includePhi1}.  You may wish to set \parlink{includePhi1} = \true~when
using \sfincs~to model an experiment, and set \parlink{includePhi1} = \false~when
comparing \sfincs~with analytic theory or with another code that does not include $\Phi_1$.

\section{Poloidal and toroidal magnetic drifts}

You can choose to either include or not include the poloidal and toroidal magnetic drifts.
These drifts are turned off by default.  To turn them on, all you need to do is
set \parlink{magneticDriftScheme} = 1 in the {\ttfamily \hyperref[sec:physicsParameters]{physicsParameters}} namelist.
(Setting \parlink{magneticDriftScheme} = 2 uses a slightly different parallel magnetic drift, which gives
indistinguishable results to setting 1 for all cases examined so far, and which is in fact exactly identical to setting 1
in the limit of vanishing plasma beta.)
If the poloidal/toroidal magnetic drifts are turned on, you must use VMEC
geometry (\parlink{geometryScheme} = 5), since the magnetic drifts
depend on various derivatives of the components of the magnetic field which
are not available in the simplified geometry models.

The magnetic drift terms introduce nonzeros in the system matrix,
and therefore increase the memory and time required for factorization.
The change is small; a typical increase in both memory and time is 20-40\%.
These magnetic drift terms typically have a minor effect on the physics outputs except when the radial
electric field is near 0.

If the electrostatic potential is not constant on flux surfaces and poloidal/toroidal magnetic drifts
are included, certain terms exist in the kinetic equation which have not yet been implemented in the code.
Thus, at present it is not strictly correct to simultaneously set  \parlink{magneticDriftScheme} $> 0$ 
and  \parlink{includePhi1} = \true.

\section{Sparse direct solver packages}
\label{sec:solvers}

As discussed briefly in section \ref{sec:gmres}, the most computationally demanding step in 
\sfincs~is the direct $LU$-factorization of a very large sparse nonsymmetric real matrix. The \PETSc~library
which \sfincs~uses has interfaces to a large number of other packages for direct factorization of such matrices,
making it possible to choose among the various solver packages with just a command-line flag ({\ttfamily -pc\_factor\_mat\_solver\_package}).  Some lists of the direct solvers
available in \PETSc~can be found \href{http://www.mcs.anl.gov/petsc/petsc-current/docs/manualpages/Mat/MatSolverPackage.html#MatSolverPackage}{here}
or in the ``direct solvers''-``LU'' section of \href{http://www.mcs.anl.gov/petsc/documentation/linearsolvertable.html}{this page}.
It is important for the $LU$-solver package to be one that is efficiently parallelized, 
in order to be able to solve problems at the high resolutions required for experimentally relevant collisionality and magnetic geometry.
The recommended choice of $LU$ solver is \href{http://mumps-solver.org/}{\mumps} (which is the default), and another good option is \href{http://crd-legacy.lbl.gov/~xiaoye/SuperLU/}{\superludist}.
(The {\ttfamily PARDISO} library available in the Intel Math Kernel Library is probably suitable as well, though we have not investigated it yet.)
In side-to-side comparisons, we find \mumps~systematically uses substantially less memory and time than \superludist~for factorization.
In principle, other solver packages for asymmetric matrices that are interfaced to \PETSc~could be used as well, such as {\ttfamily UMFPACK}, {\ttfamily PASTIX}, etc.

There are two ways to choose between solver packages.
One method is the \sfincs~parameter \parlink{whichParallelSolverToFactorPreconditioner} in the 
{\ttfamily \hyperref[sec:otherNumericalParameters]{otherNumericalParameters}} namelist.
This parameter only allows you to choose between \mumps~ and \superludist, not other solver packages interfaced to \PETSc.
Another way to choose between solver packages is the command-line flag {\ttfamily -pc\_factor\_mat\_solver\_package},
followed by one of the options in quotation marks \href{http://www.mcs.anl.gov/petsc/petsc-current/docs/manualpages/Mat/MatSolverPackage.html#MatSolverPackage}{here}.
The command-line flag overrides the namelist parameter.

The physics outputs of the code should be independent of the solver package used to several significant digits.
In principle, different solver packages solving the same linear system should find the identical solution.
However there will be small differences in the solutions associated with roundoff error.

Note that {\ttfamily superlu} and \superludist~are distinct libraries. The former is serial while the latter is parallelized.
Therefore there is no reason to use {\ttfamily superlu}; \superludist~is always preferable.

The \mumps~package has a large number of control parameters, which are documented in the \mumps~manual which can be downloaded \href{http://mumps-solver.org/}{here}.
You do not need to be aware of most of these control parameters.
However, several parameters which may be useful are discussed in section \ref{sec:mumpsControlParameters}.
It is also worth being aware of the section of the \mumps~manual on error messages. 
This section is useful for interpreting the {\ttfamily INFO(1)} and {\ttfamily INFO(2)} error codes that
are reported if \sfincs~exits with an error associated with \mumps. 

The \superludist~package has many fewer options than \mumps. 
You can find of list of the options by running \sfincs~with the {\ttfamily -help} command-line flag when
using \superludist, and searching the output for lines containing {\ttfamily superlu\_dist}.
The \superludist~options are also documented in the package's manual, available \href{http://crd-legacy.lbl.gov/~xiaoye/SuperLU/#superlu_dist}{here}.
We have not found any advantage in adjusting
any of the \superludist~options.

The \PETSc~library includes a built-in sparse direct solver which works on only a single processor.
You can select this solver using the command-line flag \\
\centerline{\ttfamily -pc\_factor\_mat\_solver\_package petsc}\\
This solver could potentially be useful if you are running on a system that does not have \mumps~or \superludist~installed,
and you are only considering problems that require sufficiently little memory (e.g. tokamaks) that parallelization is not required.
However, this solver is less robust than \mumps~or \superludist, sometimes exiting with an error message that there is a zero pivot
even though  \mumps~and \superludist~can solve the same system with no problem. Therefore, even if you plan to use only a single processor,
we still recommend that you install  \mumps~or \superludist.

\section{Parallelization: Choosing the number of nodes \& processors}
\label{sec:parallelization}

Usually the limiting factor for \sfincs~is not time but memory.
(The time required depends on the resolution used, but jobs for experimentally relevant W7-X parameters
typically take under 10 minutes.)
Therefore, when considering how many nodes to request for a \sfincs~job,
the first issue to consider is ensuring you have requested sufficient total memory.
A good way to determine the memory required (assuming you are using the default solver \mumps)
is to first run \sfincs~on 1 node using the parameters of interest
and look for the following line in standard output:\\
\centerline{\ttfamily ** TOTAL     space in MBYTES for IC factorization         :       1072}\\
The number at the end will generally be different; it depends not only on the
resolution and number of species used, but also increases with the number of processors
as discussed below.  Make sure the number of nodes requested times the number of megabytes per node exceeds this number.
This line in standard output is generated by \mumps, so if you are using a different solver, this information is not printed,
and you may need to determine the number of nodes by trial-and-error.
Since some memory on each node is used by the operating system and by \sfincs~functions other than the solver,
you may need to use a slightly higher number of nodes than this estimate suggests.
For experimentally relevant W7-X and HSX parameters, we typically use 2-6 nodes with 64 GB each.
Problems with lower resolution requirements, such as tokamaks, often can be run on a single node.

The `IC' in the above line stands for `in core', meaning the $L$ and $U$ factors are stored
in memory rather than on disk.  In \mumps, one can also choose to do an `out of core' (OOC) solve,
in which case substantially less memory is typically required.  The price you pay for this memory savings is time, since
disk access is slow compared to memory access.  Due to this slowdown, we have not used the OOC capability much, but
you might find it useful in some circumstances.
To see how much memory would be required for an OOC solve,
look for the following line in standard output:\\
\centerline{\ttfamily ** TOTAL     space in MBYTES for OOC factorization        :       128}\\
(The number at the end will generally be different.)
To invoke out-of-core mode, you must take two steps. First, use the following command-line flag when calling \sfincs:\\
\centerline{\ttfamily -mat\_mumps\_icntl\_22 1}\\
Second, you must set the environment variable {\ttfamily MUMPS\_OOC\_TMPDIR} to some reasonable directory
before calling \sfincs. This environment variable could be set for example in the batch job file.
Temporary files containing the $L$ and $U$ factors will be stored in the directory indicated.

Another issue to consider is how many processors to use.  It is not always best to use the maximum number of processes
available on the number of nodes you have chosen. The reason is that as the preconditioner matrix is divided among
more and more processes, the $LU$ factorization becomes less efficient, requiring more memory and more communication.
If you examine the \mumps~IC and OOC memory requirements indicated above, you will find they increase somewhat 
as the number of processes increase.
One needs to find a balance between speed (favoring many processes) and memory requirements (favoring few processes).
While it is almost always better overall to use 2 processes compared to 1 process, it is not always better to use 128 processes
compared to 64 processes.  The sweet spot is often in the range of 16-64 processes.
To determine how to request fewer processes than the maximum available on a given number of nodes,
see the documentation for your computing system.

If more memory is required than is available, the system will usually terminate your job with an out-of-memory (OOM) error.
When this occurs, you need to either increase the number of nodes requested or decrease the number of processors
requested.

Most of the time, \sfincs~is run via \sfincsScan~as some parameter is scanned, such as the radial electric field.
In this case, the scan is ``embarassingly parallel'' in the sense that each job in the scan is completely independent
of the other jobs.  Even if each individual job requires only 1 or a small number of nodes, it is still useful
to run \sfincs~on a computing system with many nodes so the scan can be carried out in parallel.

%\section{Issues with running on 1 processor}

\section{Monoenergetic transport coefficients}
\label{sec:monoenergetic}

By setting \parlink{RHSMode} = 3, \sfincs~can be run in a mode
where it solves the same kinetic equation (prior to discretization) as 
\dkes~and other monoenergetic codes.
When \parlink{RHSMode} = 3, the values of \parlink{Zs}, \parlink{THats}, \parlink{nHats},
\parlink{mHats}, \parlink{nu\_n}, and {\ttfamily dPhiHatdXXX} are all ignored.
Instead, the collisionality is set by \parlink{nuPrime}, and the radial electric field is set
by \parlink{EStar}.  The first of these quantities is the dimensionless collisionality
\begin{equation}
\mbox{\parlink{nuPrime}} = \frac{(G+\iota I) \nu}{v B_0}
\end{equation}
where $G$ and $I$ are defined in (\ref{eq:covariant}), $\iota=1/q$ is the rotational transform,
$v$ is the speed at which the monoenergetic calculation is being performed, and $B_0$ is the (0,0) Fourier harmonic of $B$
with respect to the Boozer poloidal and toroidal angles. The collision
frequency $\nu$ is here the value of $\nu_\mathrm{ii}$ one would have if
$v$ were the thermal speed. That is, in SI units,
\begin{equation}
  \nu=\frac{4}{3\sqrt{\pi}}\frac{n Z^4e^4\ln \Lambda}{4\pi\epsilon_0^2m^2v^3}.
\end{equation}
%
The normalized radial electric field is
\begin{equation}
\mbox{\parlink{EStar}} = \frac{cG}{\iota v B_0} \frac{d\Phi}{d\psi}
\end{equation}
(Gaussian units).
When {\ttfamily RHSMode} == 1, {\ttfamily nuPrime} and {\ttfamily EStar} are ignored.
\todo{Should be change the behavior of RHSMode=2 so it uses nuPrime and EStar instead of nu\_n?}

The two parameters \parlink{nuPrime} and \parlink{EStar} are
related to the corresponding DKES parameters {\ttfamily CMUL} and
{\ttfamily EFIELD} by
\begin{eqnarray}
  \mbox{\ttfamily
    CMUL}&\equiv&\frac{\nu_\mathrm{D}}{v}=\frac{3\sqrt{\pi}}{4}\left(\mathrm{erf}(1)-\mathrm{Ch}(1)\right)
  \frac{B_0}{G+\iota I}\mbox{\ttfamily nuPrime},\\
  \mbox{\ttfamily EFIELD}&\equiv&-\left[\frac{d\Phi}{dr}\right]_\mathrm{DKES}\frac{1}{vB_0}=-\frac{\iota}{G}\left[\frac{d\Psi}{dr}\right]_\mathrm{DKES}\mbox{\ttfamily EStar},
\end{eqnarray}
where $\mathrm{Ch}$ is the Chandrasekhar function and $\nu_\mathrm{D}$
is the actual pitch-angle deflection frequency of the particle.
\todo{These expressions are in SI units. Should there be some
  factors of $c$ to adhere to the Gaussian standard used here?
  Probably not.}

\section{Poloidal and toroidal angles}

If you are interested in any of the output quantities that vary on a flux surface,
such as the density or electrostatic potential, then it is important to know
how the poloidal and toroidal angles ($\theta$ and $\zeta$) in \sfincs~are defined.
The definitions of the poloidal and toroidal angles in 
\sfincs~depend on the input parameter \parlink{geometryScheme}. When a \vmec~equilibrium is imported by setting
\parlink{geometryScheme} = 5, then \sfincs~will use the same poloidal and toroidal angles
as \vmec.  The toroidal angle in this case is the normal cylindrical coordinate. Note that field lines
are not straight in these \vmec~coordinates.
For any other setting of \parlink{geometryScheme}, \sfincs~will use Boozer coordinates.

