\chapter{Input Parameters}

\newcommand{\param}[5]{{\setlength{\parindent}{0cm} {\ttfamily \bfseries #1}\\{\it Type}: #2\\{\it Default}: #3\\{\it When it matters}: #4\\{\it Meaning}: #5}}
\newcommand{\ssparam}[4]{{\setlength{\parindent}{0cm} {\ttfamily \bfseries #1}\\{\it Type}: #2\\{\it When it matters}: #3\\{\it Meaning}: #4}}
\newcommand{\PETScParam}[2]{{\setlength{\parindent}{0cm} {\ttfamily \bfseries #1}\\{\it Meaning}: #2}}
\newcommand{\myhrule}{{\setlength{\parindent}{0cm} \hrulefill }}

\newcommand{\true}{{\ttfamily .true.}}
\newcommand{\false}{{\ttfamily .false.}}

In this chapter we first describe all the parameters which can be included in the {\ttfamily input.namelist} file. 
Then we list some of the command-line flags associated with \PETSc~which can be useful.




\section{The {\ttfamily general} namelist}

\param{RHSMode}
{integer}
{1}
{Always}
{Option related to the number of right-hand sides (i.e. inhomogeneous drive terms) for which the kinetic equation is solved.\\

{\ttfamily RHSMode=1}: Solve for a single right-hand side.\\

{\ttfamily RHSMode=2}: Solve for 3 right-hand sides to get the 3x3 transport matrix. Presently implemented only for 1 species.\\

{\ttfamily RHSMode=3}: Solve for the 2x2 monoenergetic transport coefficients. When this option is chosen, {\ttfamily Nx} is set to 1 and only 1 species is used.
}

\myhrule

\param{outputFileName}
{string}
{``sfincsOutput.h5''}
{Always}
{Name which will be used for the HDF5 output file.  If this parameter is changed from the default value, \sfincsScan~ will not work.}

\myhrule

\param{saveMatlabOutput}
{Boolean}
{\false}
{Always}
{If this switch is set to true, Matlab m-files are created which
 store the system matrix, right-hand side, and solution vector.  If an iterative solver is used,
 the preconditioner matrix is also saved.
 PETSc usually generates an error message if you ask to save Matlab output when
the size of the linear system is more then 1400 $\times$ 1400,
 so usually this setting should be false except for very small test problems.
}

\myhrule

\param{MatlabOutputFilename}
{string}
{``sfincsMatrices''}
{Only when {\ttfamily saveMatlabOutput == .true.}}
{Start of the filenames which will be used for Matlab output.}

\myhrule

\param{saveMatricesAndVectorsInBinary}
{Boolean}
{\false}
{Always}
{If this switch is set to true, the matrix, right-hand-side, and solution of the
linear system will be saved in PETSc's binary format.  The preconditioner matrix will also
be saved if {\ttfamily tryIterativeSolver == .true.}
}

\myhrule

\param{binaryOutputFilename}
{string}
{``sfincsBinary''}
{Only when {\ttfamily saveMatricesAndVectorsInBinary == .true.}}
{Start of the filenames which will be used for binary output.}

\myhrule

\param{solveSystem}
{Boolean}
{\true}
{Always}
{If this parameter is false, the system of equations will not actually be solved.
 Sometimes it can be useful to set this parameter to {\ttfamily .false.} when debugging.
}


\section{The {\ttfamily geometryParameters} namelist}

\param{inputRadialCoordinate}
{integer}
{3}
{When {\ttfamily geometryScheme}==5, 11, or 12}
{Which radial coordinate to use to specify the flux surface for a single calculation,
or to specify the range of flux surfaces for a radial scan. See section \ref{sec:radialCoordinates}
for more information about radial coordinates.\\

{\ttfamily inputRadialCoordinate}==0: Use the flux surface specified by {\ttfamily psiHat\_wish}
for a single run, and use the range specified by {\ttfamily psiHat\_min} and {\ttfamily psiHat\_max} for radial scans.\\

{\ttfamily inputRadialCoordinate}==1: Use the flux surface specified by {\ttfamily psiN\_wish}
for a single run, and use the range specified by {\ttfamily psiN\_min} and {\ttfamily psiN\_max} for radial scans.\\

{\ttfamily inputRadialCoordinate}==2: Use the flux surface specified by {\ttfamily rHat\_wish}
for a single run, and use the range specified by {\ttfamily rHat\_min} and {\ttfamily rHat\_max} for radial scans.\\

{\ttfamily inputRadialCoordinate}==3: Use the flux surface specified by {\ttfamily rN\_wish}
for a single run, and use the range specified by {\ttfamily rN\_min} and {\ttfamily rN\_max} for radial scans.
}


\section{Directives for \sfincsScan}
\label{sec:sfincsScanParams}

The parameters for \sfincsScan~ begin with the code {\ttfamily !ss} and so are not read by the fortran part of \sfincs.
These parameters matter only when \sfincsScan~is called and are all ignored when \sfincs~is executed directly.
These parameters can appear anywhere in the {\ttfamily input.namelist} file, in any namelist or outside of any namelist.
Note that \sfincsScan~ parameters do not have defaults, unlike fortran namelist parameters.\\

\ssparam{scanType}
{integer}
{Any time \sfincsScan~is called.}
{Which type of scan will be run when \sfincsScan~is called. \\

{\ttfamily scanType=1}:  Convergence scan. (Scan the parameters in the resolutionParameters namelist.)\\

{\ttfamily scanType=2}:  Scan of $E_r$.\\

{\ttfamily scanType=3}:  Scan any one input parameter that takes a numeric value.\\

{\ttfamily scanType=4}:  Scan radius, taking the density and temperature profiles from the {\ttfamily profiles} file.
In this type of scan, the same radial electric field is used at every radius.
See {\ttfamily utils/profiles.XXX} for examples.\\

{\ttfamily scanType=5}:  Scan radius, and at each radius, scan $E_r$.  Density and temperature profiles are again
 taken from the {\ttfamily profiles} file; see {\ttfamily utils/profiles.XXX} for examples.\\

{\ttfamily scanType=21}: Read in a list of requested runs from a file {\ttfamily runspec.dat}. See {\ttfamily utils/sfincsScan\_21} for an example file.
}

\myhrule


\section{\PETSc~commands}

Command-line flags can be used to modify the behavior of any \PETSc~application, including \sfincs.
There are hundreds of \PETSc~options, and a list can be obtained by running with the command-line flag
{\ttfamily -help}. Here we list some of the more useful options.\\

\PETScParam{-help}
{Dumps a list of available command-line options to stdout.}

\myhrule

\PETScParam{-ksp\_view}
{Dumps detailed information to stdout related to the linear solver.}


